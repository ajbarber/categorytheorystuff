\documentclass{article}
\usepackage{amsmath}
\usepackage{dsfont}
\usepackage{tikz-cd}
\usetikzlibrary{matrix}
\begin{document}

Leitner - Basic Category Theory - Problem solutions
Adam Barber


\textbf{0.10:} \\

Let $S$ be a set. The indiscrete topological space $I(S)$ is the space whose set of points is S and whose only open subsets are $\emptyset$ and $S$.
To find a universal property satisfied by the space $I(S)$ proceed as follows.
With this topology any map from a topological space to S is continuous. \\

Parroting the wording of the question, let us rephrase this in
universal parlance. Define a function $i: S \rightarrow I(S)$, by $i(s) = s, s \in S$.
Then $I(S)$ has the following property.

% \begin{tikzcd}
%   S \arrow[r,"i"]  \arrow[d,"G"{name=G,right}]& I(S) \\
%   Z \arrow[ur, "H"{name=H,right}, dashrightarrow]
% \end{tikzcd}

\begin{center}
\begin{tikzcd}
  S \arrow[r, "i"] & I(S) \\
& X \arrow[lu, "\forall f"{name=H,below}, dashrightarrow] \arrow[u,"\overline{f}"{name=G,left}]
\end{tikzcd}
\end{center}

For all topological spaces $X$ and all functions
$f: X \rightarrow S$ there exists a unique continuous map $\overline{f}: X \rightarrow I(S)$. What it says is all maps into an indiscrete space are continuous. It also says that given $S$, the universal property determines $I(S)$ and $i$, up to isomorphism. \\

\textbf{0.11} \\
The universal property that is satisfied by the pair $(ker(\theta),\iota)$ is depicted in the diagram below.

\begin{center}
\begin{tikzcd}
  ker(\theta) \arrow[r, "\iota"{name=iota, above}] &
  G \arrow[r, "\epsilon"{name=eps, above}, shift right = 1.5ex]
    \arrow[r, "\theta"{name=theta,below}] & H  \\
  F \arrow[u, "\exists! \overline{f}"{name=f,left}, dashrightarrow] \arrow[ru,"\forall f"{name=i,below}]
\end{tikzcd}
\end{center}



\textbf{0.13}:

(a)

Choose $\phi(\sum_{i=1}^na_ix^i) = \sum_{i=1}^na_ir^i$. Then $\phi$ with $\phi(x)=r$ is a homomorphism that
satisfies additive and multiplicative properties. To prove uniqueness assume there is another
homomorphism $\psi$, with $\psi(x)=r$. Then $\psi(\sum_{i=1}^na_ix^i) = \sum_{i=1}a_i\psi(x) = \sum_{i=1}a_ir^i$ by properties of
a homomorphism. So $\psi=\phi$. \par


(b)

$\iota \colon \mathds{Z}[x] \rightarrow A$ maps
$\sum_{i=1}^n p_ix^i$ to $\sum_{i=1}^n p_ia^i$, using $\iota(x) = a$,
the multiplicative property of a homomorphism to get $\iota(x^i)=\iota(x)^i$,
and the additive property to get $\iota(p_i)\iota(x)^i = p_i\iota(x)^i$ remembering $p_i$ is in $\mathds{Z}$. \par

Going in the direction $A \rightarrow \mathds{Z}[x]$ we know as provided in (b) that,
taking $R=\mathds{Z}[x]$, and $\phi=\iota^\prime$, there exists a unique ring homomorphism such that $\iota^\prime(a)=x$.
So $\iota^\prime $ maps $\sum_{i=1}^np_ia^i$ to $\sum_{i=1}^{n}p_ix^i$ and $\iota^\prime \circ \iota = 1_{\mathds{Z}[x]}$.
Also using definitions of $\iota$ and $\iota^\prime$ easily yields $\iota \circ \iota^\prime = 1_A$.


\end{document}