\documentclass{article}
\usepackage{amsmath}
\usepackage{dsfont}
\usepackage{tikz-cd}
\usetikzlibrary{matrix}
\usepackage{lipsum}
\usepackage{titlesec}
\usepackage{parskip}
\begin{document}

\title{Leinster - Basic Category Theory - Selected problem solutions}
\author{Adam Barber}

\maketitle
\subsubsection*{0.10}

Let $S$ be a set. The indiscrete topological space $I(S)$ is the space whose set of points is S and whose only open subsets are $\emptyset$ and $S$.
To find a universal property satisfied by the space $I(S)$ proceed as follows.
With this topology any map from a topological space to S is continuous.

Parroting the wording of the question, let us rephrase this in
universal parlance. Define a function $i\colon S \rightarrow I(S)$, by $i(s) = s, s \in S$.
Then $I(S)$ has the following property.

% \begin{tikzcd}
%   S \arrow[r,"i"]  \arrow[d,"G"{name=G,right}]& I(S) \\
%   Z \arrow[ur, "H"{name=H,right}, dashrightarrow]
% \end{tikzcd}

\begin{center}
\begin{tikzcd}
  S \arrow[r, "i"] & I(S) \\
& X \arrow[lu, "\forall f"{name=H,below}, dashrightarrow] \arrow[u,"\bar{f}"{name=G,left}]
\end{tikzcd}
\end{center}

For all topological spaces $X$ and all functions
$f\colon X \rightarrow S$ there exists a unique continuous map $\bar{f}\colon X \rightarrow I(S)$. What it says is all maps into an indiscrete space are continuous. It also says that given $S$, the universal property determines $I(S)$ and $i$, up to isomorphism.

\subsubsection*{0.11}

The universal property that is satisfied by the pair $(ker(\theta),\iota)$ is depicted in the diagram below.

\begin{center}
\begin{tikzcd}
  ker(\theta) \arrow[r, "\iota"{name=iota, above}] &
  G \arrow[r, "\epsilon"{name=eps, below}, shift right = 1.5ex]
    \arrow[r, "\theta"{name=theta, above}] & H  \\
  F \arrow[u, "\exists! \bar{f}"{name=f,left}, dashrightarrow] \arrow[ru,"\forall f"{name=i,below}]
\end{tikzcd}
\end{center}

The statement of the universal property is as follows. For any $f\colon F \rightarrow G$ such that $\theta \circ f = \epsilon \circ f$, there is a unique $\bar{f}\colon F \rightarrow ker(\theta)$ such that the diagram above commutes. That is $f = \iota \circ \bar{f}$. \\

\subsubsection*{0.13}
\paragraph{(a)}
Choose $\phi(\sum_{i=1}^na_ix^i) = \sum_{i=1}^na_ir^i$. Then $\phi$ with $\phi(x)=r$ is a homomorphism that
satisfies additive and multiplicative properties. To prove uniqueness assume there is another
homomorphism $\psi$, with $\psi(x)=r$. Then $\psi(\sum_{i=1}^na_ix^i) = \sum_{i=1}a_i\psi(x) = \sum_{i=1}a_ir^i$ by properties of
a homomorphism. So $\psi=\phi$. \par

\paragraph{(b)}
$\iota \colon \mathds{Z}[x] \rightarrow A$ maps
$\sum_{i=1}^n p_ix^i$ to $\sum_{i=1}^n p_ia^i$, using $\iota(x) = a$,
the multiplicative property of a homomorphism to get $\iota(x^i)=\iota(x)^i$,
and the additive property to get $\iota(p_i)\iota(x)^i = p_i\iota(x)^i$ remembering $p_i$ is in $\mathds{Z}$. \par

Going in the direction $A \rightarrow \mathds{Z}[x]$ we know as provided in (b) that,
taking $R=\mathds{Z}[x]$, and $\phi=\iota^\prime$, there exists a unique ring homomorphism such that $\iota^\prime(a)=x$.
So $\iota^\prime $ maps $\sum_{i=1}^np_ia^i$ to $\sum_{i=1}^{n}p_ix^i$ and $\iota^\prime \circ \iota = 1_{\mathds{Z}[x]}$.
Also using definitions of $\iota$ and $\iota^\prime$ easily yields $\iota \circ \iota^\prime = 1_A$.

\subsubsection*{0.14}
\paragraph{(a)}

For the triangles below to commute, we need, as stated in the question $p_1 \circ f = f_1$ and $p_2 \circ f = f_2$.\\
\begin{center}

\begin{tikzcd}
  V \arrow[r, "f"] \arrow[dr, "\forall f_{1}"{name=H,below}, dashrightarrow] & P \arrow[d,"\exists p_{1}"{name=G,right}] \\
& X
\end{tikzcd} \break\hfill

\begin{tikzcd}
  V \arrow[r, "f"] \arrow[dr, "\forall f_{2}"{name=H,below}, dashrightarrow] & P \arrow[d,"\exists p_{2}"{name=G,right}] \\
& Y
\end{tikzcd}
\end{center}
Choosing $P=X\times{Y}$, $p_1$ and $p_2$ as below makes the triangles commute.
\begin{center}
$p_1\colon X\times{Y} \rightarrow X$ \\
$p_2\colon X\times{Y} \rightarrow Y$
\end{center}

\paragraph{(b)}

Proving uniqueness involves taking two arbitrary cones with the property stated in (a).
Taking $(P, p_1, p_2)$ and $(P^\prime, p_1^\prime, p_2^\prime)$
we know from (a) that for all cones $(V, f_1, f_2)$ there exists a unique linear map $f\colon V \rightarrow P^\prime$ such that $p_1^\prime \circ f = f_1$, $p_2^\prime \circ f = f_2$.
In this statement choose $V=P^\prime$, then referring to the triangles in (a), observe there exists a $f\colon P \rightarrow P^\prime$ such that $p_1^\prime \circ f = p_1$, $p_2^\prime \circ f = p_2$.

\textbf{Comment} The choice of P and p notation hinted very heavily that this is a projection of a product.

\paragraph{(c)}

We need to define the cocone $(Q, q_1, q_2)$ with the property, for all cocones $(V, f_1, f_2)$ there exists a unique linear map $f\colon Q \rightarrow V$ such that $f \circ q_1 = f_1$ and $f \circ q_2 = f_2$. Choose $Q= X \times Y$, $q_1\colon X \rightarrow X \oplus Y$, $q_2\colon Y \rightarrow X \oplus Y$. \\

\textbf{Comment} This is the dual of the product in (b), the coproduct. Set equivalent is the disjoint union.

\subsubsection*{1.2.22}

Given that A and B are preordered sets, we know that, taking for instance $a \in A$, the relation $\leq$ is reflexive, $a \leq a$ for all $a \in A$, and transitive, $a \leq b$ and $b \leq c$ then $a \leq c$, for all $a, b, c \in A$.  Taking $\leq$ as the morphism for our category, we can immediately see that the reflexive property of a preorder is equivalent to the identity mapping requirement in a category. And the transitivity property of a preorder demonstrates the associativity of morphisms requirement of a category. So $\mathcal{A}$ and $\mathcal{B}$ are categories. Given a functor $F: f \mapsto F(f)$ then, this functor sends $a \leq b$ to $f(a) \leq f(b)$

\subsubsection*{1.2.25}

\paragraph{(a)}

Let $F: \mathcal{A} \times \mathcal{B} \rightarrow \mathcal{C}$ be a functor. We are given that for each $A$ in $\mathcal{A}$ there is a morphism $F^A: \mathcal{B} \rightarrow \mathcal{C}$ defined on objects $B$ in $\mathcal{B}$ by $F^A = F(A,B)$ and on maps $g$ in $\mathcal{B}$  by $F_A(g) = F(1_A, g)$. We need to prove $F^A$ is a functor.\\

First, we need to show  $F^A(g \circ \bar{g}) = F^A(g) \circ F^A(\bar{g})$.

\begin{align*}
  F^A(g \circ \bar{g})&= F(1_A, g \circ \bar{g})  \\
                           &= F(1_A,g) \circ F(1_A,\bar{g}) \\
                           &= F^A(g) \circ F^A(\bar{g})
\end{align*}

The second step above uses our formula from composition of a product category derived Ex 1.1.14. \\

We also need
\begin{align*}
F^A(1_B) &= F(1_A, 1_B) = 1_C
\end{align*}

The identity maps because $F$ is a functor $\mathcal{A} \times \mathcal{B} \rightarrow \mathcal{C}$. So $F^A$ is a functor.

Apply analogous reasoning for $F_B$.
\paragraph{(b)}

We are given $F: \mathcal{A} \times \mathcal{B} \rightarrow \mathcal{C}$ is a functor.
The question asks us to show for all $A \in \mathcal{A}$ and all $B \in \mathcal{B}$

\begin{align}
\label{eqn:1225b1}
  F^A(B) &= F_B(A)
\end{align}

and if $f: A \rightarrow A^{\prime}$ in $\mathcal{A}$ and $g: B \rightarrow B^\prime$ in
$\mathcal{B}$ then
\begin{align*}
F^{A^{\prime}}(g) \circ F_B(f) = F_{B^\prime}(f) \circ F^A(g). \\
\end{align*}
In the following answers recall that:
\begin{align*}
  F^A(g) = F(1_A, g) \\
  F_B(f) = F(1_B, f)
\end{align*}
Equation (\ref{eqn:1225b1}) is verified by basic checking. Consider the second equation above along with the diagram below.

\begin{center}
\begin{tikzcd}
  & (A,B) \arrow[d,"F^A(g)"] \arrow[r, "F_B(f)"] & (A',B) \arrow[d,"F^{A^{\prime}}(g)"{name=G,right}]\\
  & (A,B') \arrow[r,"F_{B^{\prime}}(f)"] & (A',B')
\end{tikzcd}
\end{center}

We know from Exercise 1.1.14 that in the product category represented by $\mathcal{A} \times \mathcal{B}$, maps compose in the following manner
\begin{align*}
  (f, g) \circ (f^{\prime},g^{\prime}) = (ff^{\prime}, gg^{\prime})
\end {align*}

We also know from the axioms of our functor $F$ that different strings of maps under $F$ from $F(A,B)$ to $F(A',B')$ are equal.\footnote{See Remarks 1.2.2 of the Leitner text} So the above square commutes and we have the required equality
\begin{align*}
  F^{A^{\prime}}(g) \circ F_B(f) = F_{B^\prime}(f) \circ F^A(g)
\end{align*}

\paragraph{(c)}

We need to prove there is a unique functor $F$, satisfying the conditions in (a.). Take families of functors $F^A$ and $F_B$ as in (b), which satisfy the below

\begin{itemize}
\item $ \text{If }f\colon A \rightarrow A^{\prime} \text{ in } \mathcal{A}, g\colon B \rightarrow B^{\prime} \text{ in } \mathcal{B}, \text{ then } F^{A^{\prime}}(g) \circ F_B(f) = F_{B^\prime}(f) \circ F^A(g)$
\item $F^A(B) = F_B(A) \text{ if } A \in \mathcal{A}, B \in \mathcal{B},$
\end{itemize}

To begin, write
\begin{align*}
  F &= F^A(g) \circ F_B(f) \text{ for morphisms, } \\
  F & =F^A(B) \text{ for objects. } \\
    & =F_B(A) \\
    & = F(A,B)
\end{align*}

We need to prove that $F$ is a functor. We are given in this question that $F_A$, $A \in \mathcal{A}$ and $F_B$, $B \in \mathcal{B}$ are functors.
\begin{align*}
  F(f \circ \bar{f}, g \circ \bar{g}) &= F_A(g \circ \bar{g}) \circ F_B(f \circ \bar{f}) \\
                                                &= F_A(g) \circ F_A(\bar{g}) \circ F_B(f) \circ F_B(\bar{f}) \\
                                                &= F_A(g) \circ F_{B^{\prime}}(f) \circ F_A(\bar{g}) \circ F_B(f) \text{ using result from (b.)} \\
                                      &= F(f,g) \circ F(\bar{f}, \bar{g}) \\ \\
  \text { Also, } \\\\
  F(1_A,1_B) &= F_A(1_A) \circ F_B(1_B) \\
             &= 1_{F_A(A)} \circ 1_{F_B(B)}
\end{align*}

So functions compose under the functor $F$, the identity maps, and all objects are mapped. So we have established $F$ exists and is a functor. We still need to determine uniqueness. Given the property in (a)
\begin{align*}
  F^A(B) &= F_B(A) \\
         &= F(A,B)
\end{align*}
Fixing an $A \in \mathcal{A}, B \in \mathcal{B}$ our functor maps the object $(A,B)$ to $F(A,B)$. In fact for each object mapping there is a unique $F^A$ and $F_B$ that 'interlock' to produce $F(A,B)$. Put another way, fix an object in $\mathcal{C}, F(A,B)$. Then out of our two families of functors $(F^A)_{A\in\mathcal{A}}$, $(F_B)_{B\in\mathcal{B})}$ there is only one choice in each family to yield the desired object $F(A,B)$.

\subsubsection*{1.2.28}

\paragraph{(a)}

The question asks of all the functors listed in 1.2, which are faithful, and which are full.

\paragraph{Forgetful functors that forget structure}

These are typically faithful but not full.

An example in the text is $U\colon\textbf{Grp} \rightarrow \textbf{Set}$

$U$ forgets the group structure of groups and forgets that group homomorphisms are homomorphisms.

\begin{itemize}
\item To prove faithful we need to show that for one map between two objects in the source category, there is one corresponding map in the destination category. Although a group homomorphism is \textbf{not} necessarily injective, this does not concern us here. As we wish to show, \textbf{given one morphism between objects} in the source category, that this induces in the destination category at most one morphism between objects there. Then we can say the functor, here $U(f)$ is faithful. Since $U(f)$ is simply the group homomorphism $f\colon G \rightarrow H$ itself, then it is one to one for each morphism between two objects in G.

\item To prove a functor is full, we need to show for one morphism between objects in the destination category, there is one corresponding map in the source category. Set though could potentitally have more morphisms between its objects, which are \textbf{not} group homomorphisms, so picking one of these morphisms in $\textbf{Set}$, there is no corresponding morphism in $\textbf{Grp}$, in this case. So the functor is not full.
\end{itemize}

\paragraph{Forgetful functors that forget properties.}

These are typically full and faithful.

The example given is $U:\textbf{Ab} \rightarrow \textbf{Grp}$, which forgets that Abelian groups are Abelian. Here the commutativity property of the \textbf{objects} in the category is forgotten. The morphisms are unchanged. When one reads about structure, this appears to refer principallly to morphisms.

\paragraph{Free functors.}

The free construction of a group $F(S)$ is obtained from the set S, by adding just enough new elements that it becomes a group, but without imposing any other equations other than those forced by the definition of a group. The construction given in Example 1.2.3, gives an explicit formula for the functor. Indeed $S, S^\prime$ maps $f\colon S \rightarrow S^\prime$ to $F(f)\colon F(S) \rightarrow F(S^\prime)$. So as we have an explict formula which yields a $F(f)$ for a given $f$ between two objects $S, S^\prime$, allowing us to conclude the free functor here is faithful.

To refute full for the free $\textbf{Set} \rightarrow \textbf{Grp}$ construction. Take $\mathbf{Set}=\{ 1 \}$. The free functor maps this to $\mathds{Z}$. Then take in $\mathds{Z}, F(f)\colon n \mapsto 2n$. There is no corresponding $f$ on $\mathbf{Set}=\{ 1 \}$ to induce $F(f)$.

\paragraph{Group - one object category}

Let $G$ and $H$ be groups. They are single object categories. So the object refers to the group. The morphisms in the one object category are the elements of G! In this example the morphisms are not functions. Compose in the category is the binary operation on the group.

A functor from $G$ to $H$ is just a group homomorphism. For this functor to be faithful we require there is at most one morphism in the source category that induces a morphism in the destination category. Replacing the word morphism here with element of G and we have the defintion of injective. So if the group homomorphism in injective the functor is faithful. Similar reasoning tells us if the group homomorphism is surjective, then the functor is full.

\paragraph{Hom functor over vector spaces}

$\mathbf{Hom}(-,W)$ sends $f \mapsto f^*$, where $ f^*(q) = q \circ f$. In our source category we have $f\colon V \rightarrow V^\prime$. Set $W=k$ so that $q$ maps $V^\prime$ to $k$.

\end{document}