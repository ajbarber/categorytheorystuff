\documentclass{article}
\usepackage{amsmath}
\usepackage{dsfont}
\usepackage{tikz-cd}
\usetikzlibrary{matrix}
\usepackage{lipsum}
\usepackage{titlesec}
\usepackage{parskip}
\begin{document}

\title{Leinster - Basic Category Theory - Selected problem solutions for Chapter 3}
\author{Adam Barber}

\maketitle

\subsubsection*{3.1.1}

There are bijections

\begin{align*}
  (A + B, C) &\leftrightarrow ((A,B), \Delta C) \\
  f &\leftrightarrow \overline{f}
\end{align*}

where $\overline{f} = (f,f)$

\begin{align*}
  (\Delta A, (B, C)) &\leftrightarrow (A, B \times C) \\
  g = (p,q) &\leftrightarrow \overline{g}
\end{align*}

where $\overline{g}(x) = (p(x), q(x))$

So the sum is left adjoint to $\Delta$, and the product is its right adjoint.

\subsubsection*{3.1.2}

We are given the definition of a sequence, where there is a unique function $x$ such that the square below commutes.

We have $x_0 = a$, and $x_{n+1} = r(x_n)$.

\begin{center}
\begin{tikzcd}
  & \mathds{N} \arrow[d,"x"] \arrow[r, "s"] & \mathds{N} \arrow[d,"x"{name=G,right}]\\
  & X \arrow[r,"r"] & X
\end{tikzcd}
\end{center}

This is precisely the definition of the comma category $(\mathds{N} \Rightarrow X)$, where objects are $(n \in \mathds{N}, x, t \in X)$.

\subsubsection*{3.2.12}
\paragraph{(a)}

\begin{align*}
  \theta(S) = \bigcup \theta(R) \supseteq \bigcup R = S
\end{align*}

But $\theta^2(S) = \theta(S)$, so $\theta(S) \subseteq S$.

Taken together, the above implies $\theta(S) = S$.

\paragraph{(b)}
\begin{align*}
  A &\subseteq B \\
  \implies f\left(A\right) &\subseteq f\left(B\right) \\
  \implies gf\left(A\right) &\subseteq gf\left(B\right) \\
\end{align*}

$g$ and $f$ are taken to be injections here. We need to prove there is a bijection between $A$ and $B$. \textbf{Note:} this does not follow immediately from $g$
and $f$ being injections.

Take $\theta(S) = A - g\left(B \setminus f\left(S\right)\right)$. Then $S_1 \subseteq S_2 \implies \theta(S_1) \subseteq \theta(S_2)$. Since $f, g$ and hence $\theta$ is order preserving, we may apply the result in (a). Specifically, there exists $S$ such that $S = A - g\left(B \setminus f\left(S\right)\right) \implies g\left(B \setminus f\left(S\right)\right) = A \setminus S$.

\paragraph{(c)}

We need to prove a bijection between $A$ and $B$ to deduce the theorem. Consider $h\colon A \rightarrow B$

\begin{align*}
  h(x) &= \begin{cases}
    f(x), & x \in S,  \\
    g^{-1}(x), & x \in A \setminus S
  \end{cases} \\
\end{align*}

$f$ has a codomain of $f\left[S\right]$, so every element of the codomain has a preimage in $S$. We are given that $f$ is injective.

$g$ is injective and hence invertible. Using the result in (b) we have a direct expression for $g^{-1}$. Hence we have $gh = 1_A$, and $hg = 1_B$, for $x$ in $A \setminus S$.

An alternative proof, has a similar basic idea, of partitioning the domain of the bijection around the fixed point. \textbf{Sketch proof} Set $A_0 = A$. $A_{i+1} = gfA_{i}$. Define $k(x) = gf(x)$ if $x \in A_{i}$ for some $i$, otherwise $k(x) = x$. To prove $k$ is surjective comes down to two cases. Suppose $y \in A_{n}$, for some $n$ then $A_{n-1}$ is the $x$-value such that $k(x) = y$. If $y$ is not in $A_{n}$ for any arbitrarily large $n$, then we must have $k(x) = x$.

\subsubsection*{3.2.14}

Need to prove that for any family $(A_i)_{i `\in I}$  of objects of $\mathcal{A}$, there is some object of $\mathcal{A}$ not isomorphic to $A_i$ for $i \in I$.
It suffices to prove for $A$ in $F(S)$,  $F: \mathbf{Set} \rightarrow \mathcal{A}$, then we know the condition holds for $\mathcal{A}$. Now $UF$ is injective by Exercise 2.3.11,
so $U$ is injective on objects $A$ of $F(S)$. So if $UA_i$ is not isomorphic to $UA_j$, this would imply $A_i$ is not isomorphic to $A_j$. So we need to prove for a given $i$,  $|UA_i| < |\mathcal{P}(UA)|$:

\begin{equation*}
  |UA_i| \leq |\Sigma UA_i| < |\mathcal{P}(UA)|
\end{equation*}

The strict equality due to Theorem 3.2.2.

\subsubsection*{3.2.15}

The key point here is that $Set$ is not small. I think of $Set$ as a power set of an arbitrary family of sets, as in the proof for Proposition 3.2.4. Set is locally small however, as for any two objects $A$ and $B$, the functions between $A$ and $B$ form a set. This question is a little too wooly for me, I struggled, without the necessary background, to reason my way though so many ambiguities that presented themselves. Here is a shot.

\paragraph{(a)}
\textbf{Mon} is equivalent to a single object category, which is small. So \textbf{Mon} is essentially small.
\paragraph{(b)}
$\mathds{Z}$, the group of integers viewed as a one object category, is locally small. Groups are just an 'enriched' set.
\paragraph{(c)}
The ordered set of integers still has a large class of isomorphism classes (?) My guess here is it locally small, as there is one map between each two objects.
\paragraph{(d)} Using the existence of a left adjoint proved in 3.2.16, and the result of 3.2.14, tells us the class of isomorphism classes of \textbf{Cat} is large. So \textbf{Cat} is not essentially small. For locally small we would require the set of natural transformations between \textbf{Cat} and \textbf{Set} be a set. There is one component for each object in $\textbf{Cat}$ which is small, hence the morphisms form a single element set. $Cat$ is locally small. (?)
\paragraph{(e)} \textbf{Guess}. Same reasoning as (a), locally small.

\end {document}
