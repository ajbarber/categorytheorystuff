\documentclass{article}
\usepackage{amsmath}
\usepackage{dsfont}
\usepackage{tikz-cd}
\usetikzlibrary{matrix}
\usepackage{lipsum}
\usepackage{titlesec}
\usepackage{parskip}
\begin{document}

\title{Leinster - Basic Category Theory - Selected problem solutions for Chapter 3}
\author{Adam Barber}

\maketitle

\subsubsection*{3.1.1}

There are bijections

\begin{align*}
  (A + B, C) &\leftrightarrow ((A,B), \Delta C) \\
  f &\leftrightarrow \overline{f}
\end{align*}

where $\overline{f} = (f,f)$

\begin{align*}
  (\Delta A, (B, C)) &\leftrightarrow (A, B \times C) \\
  g = (p,q) &\leftrightarrow \overline{g}
\end{align*}

where $\overline{g}(x) = (p(x), q(x))$

So the sum is left adjoint to $\Delta$, and the product is its right adjoint.

\subsubsection*{3.1.2}

$x_0 = a$, and $x_{n+1} = r(x_n)$.

We are given the definition of a sequence, where there is a unique function $x$ such that the square below commutes.

\begin{center}
\begin{tikzcd}
  & \mathds{N} \arrow[d,"x"] \arrow[r, "s"] & \mathds{N} \arrow[d,"x"{name=G,right}]\\
  & X \arrow[r,"r"] & X
\end{tikzcd}
\end{center}

This is precisely the definition of the comma category $(\mathds{N} \Rightarrow X)$, where objects are $(n \in \mathds{N}, x, t \in X)$.

\end {document}
