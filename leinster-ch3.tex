\documentclass{article}
\usepackage{amsmath}
\usepackage{dsfont}
\usepackage{tikz-cd}
\usetikzlibrary{matrix}
\usepackage{lipsum}
\usepackage{titlesec}
\usepackage{parskip}
\begin{document}

\title{Leinster - Basic Category Theory - Selected problem solutions for Chapter 3}
\author{Adam Barber}

\maketitle

\subsubsection*{3.1.1}

There are bijections

\begin{align*}
  (A + B, C) &\leftrightarrow ((A,B), \Delta C) \\
  f &\leftrightarrow \overline{f}
\end{align*}

where $\overline{f} = (f,f)$

\begin{align*}
  (\Delta A, (B, C)) &\leftrightarrow (A, B \times C) \\
  g = (p,q) &\leftrightarrow \overline{g}
\end{align*}

where $\overline{g}(x) = (p(x), q(x))$

So the sum is left adjoint to $\Delta$, and the product is its right adjoint.

\subsubsection*{3.1.2}

We are given the definition of a sequence, where there is a unique function $x$ such that the square below commutes.

We have $x_0 = a$, and $x_{n+1} = r(x_n)$.

\begin{center}
\begin{tikzcd}
  & \mathds{N} \arrow[d,"x"] \arrow[r, "s"] & \mathds{N} \arrow[d,"x"{name=G,right}]\\
  & X \arrow[r,"r"] & X
\end{tikzcd}
\end{center}

This is precisely the definition of the comma category $(\mathds{N} \Rightarrow X)$, where objects are $(n \in \mathds{N}, x, t \in X)$.

\subsubsection*{3.2.12}
\paragraph{(a)}

\begin{align*}
  \theta(S) = \bigcup \theta(R) \supseteq \bigcup R = S
\end{align*}

But $\theta^2(S) = \theta(S)$, so $\theta(S) \subseteq S$.

Taken together, the above implies $\theta(S) = S$.

\paragraph{(b)}
\begin{align*}
  A &\subseteq B \\
  \implies fA &\subseteq fB \\
  \implies gfA &\subseteq gfB \\
\end{align*}

$g$ and $f$ are taken to be injections here. We need to prove there is a bijection between $A$ and $B$. \textbf{Note:} this does not follow immediately from $g$
and $f$ being injections.

So $gf$ is order preserving. Which means by (a) there is some $gf(S) = S$, $S \in A$. So we know how the bijection behaves on $S$, but we need to specify its behaviour for $A \setminus S$. Set $g(B) = A$. Then $g(B) - S = g(B) - gf(S) = g(B \setminus fS) = A \setminus S$. (*).

\paragraph{(c)}

We need to prove surjectivity of $g$ to deduce the theorem. So for all $a \in A$ we need some $b$ such that $g(b) = a$. For $a \in S$ this holds immediately since $gf(S) = S$. For $a \in A \setminus S$, we have the equation for $g$ derived in (b).

\subsubsection*{3.2.14}

Need to prove that for any family $(A_i)_{i `\in I}$  of objects of $\mathcal{A}$, there is some object of $\mathcal{A}$ not isomorphic to $A_i$ for $i \in I$.
It suffices to prove for $A$ in $F(S)$,  $F: \mathbf{Set} \rightarrow \mathcal{A}$, then we know the condition holds for $\mathcal{A}$. Now $UF$ is injective by Exercise 2.3.11,
so $U$ is injective on objects $A$ of $F(S)$. So if $UA_i$ is not isomorphic to $UA_j$, this would imply $A_i$ is not isomorphic to $A_j$. So we need to prove for a given $i$,  $|UA_i| < |\mathcal{P}(UA)|$:

\begin{equation*}
  |UA_i| \leq |\Sigma UA_i| < |\mathcal{P}(UA)|
\end{equation*}

The strict equality due to Theorem 3.2.2.

\end {document}
