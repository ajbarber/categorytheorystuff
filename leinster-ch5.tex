\documentclass{article}
\usepackage{amsmath}
\usepackage{amssymb}
\usepackage{dsfont}
\usepackage{tikz-cd}
\usetikzlibrary{matrix}
\usepackage{lipsum}
\usepackage{titlesec}
\usepackage{parskip}
\begin{document}

\title{Leinster - Basic Category Theory - Selected problem solutions for Chapter 5}
\author{Adam Barber}

\maketitle

\subsubsection*{5.1.34}

The equaliser square is not necessarily a pullback. There is no reason why any function into the $X$ would commute with a unique function into $E$, composed with $i$.

The converse is true though, a pullback implies an equaliser, when the square is set up as in the question.

\subsubsection*{5.1.35}

Suppose the right hand square is a pullback. We need to prove the left hand square is a pullback if and only if the full rectangle, which composes both squares, is a pullback.

\paragraph{Only if}

Assume the left hand square and right hand squares are pullbacks. Show full rectangle is a pullback, that is show $g_1 = p_{2\cdot}p_{12}\overline{f}$, and $f_2 = p_{1\cdot}\overline{f}$

\begin{equation*}
\begin{tikzcd}
A
\arrow[drr, bend left, "f_1"]
\arrow[drr, "\overline{g}", dashed, yshift=0.2ex, red]
\arrow[ddr, bend right, "f_2"]
\arrow[dr, dashed, "\overline{f}" description]
\arrow[drrr, bend left=50, dotted, red, yshift=0.3ex, "g_1"]
\arrow[ddrr, bend right=90, dotted, red, yshift=-0.8ex, "g_2"'] & & \\
& P_1 \arrow[r, "p_{12}"] \arrow[d, "p_{1\cdot}"] & P_2 \arrow[r, "p_{2\cdot}", red] \arrow[d, "", red]
& \cdot \arrow[d, "" , red] \\
& \cdot \arrow[r, ""] & \cdot \arrow[r, "", red]
& \cdot
\end{tikzcd}
\end{equation*}

\textbf{Left square pullback (black):} For any $f_1$ and $f_2$, there is a unique map $\overline{f}$ such that the left square above commutes.

\textbf{Right square pullback (red):}  For any $g_1$ and $g_2$, there is a unique $\overline{g}$ such that the red diagram commutes.

Due to the left hand square being a pullback, for each $f_1$, and $f_2$, there is a unique map $\overline{f}$ such that $f_2 = p_{1\cdot}\overline{f}$ and $f_1 = p_{12}\overline{f}$. Set $f_1 = \overline{g}$.   From the right hand side being a pullback,  $g_1 = p_{2\cdot}\overline{g}=p_{2\cdot}p_{12}\overline{f}$ as required.

\paragraph{If}

Assume the outer rectangle and right hand square are both pullbacks. Show the left hand side square is a pullback, that is $f_1 = p_{12}\overline{h}$, and $h_2=p_{1\cdot}\overline{h}$, for any $f_1,  h_2$.

\begin{equation*}
\begin{tikzcd}
A
\arrow[drrr, bend left, "h_1"]
\arrow[drr, "\overline{g}", yshift=0.2ex, dotted, red]
\arrow[ddr, bend right, "h_2"]
\arrow[dr, dotted, "\overline{h}" description]
\arrow[drrr, bend left=50, dotted, red, yshift=0.5ex, "g_1"]
\arrow[ddrr, bend right=90, dotted, red, yshift=-0.8ex, "g_2"'] & & \\
& P_1 \arrow[r, "p_{12}"] \arrow[d, "p{1\cdot}"] & P_2 \arrow[r, "p_{2\cdot}"] \arrow[d, ""]
& \cdot \arrow[d, ""] \\
& \cdot \arrow[r, ""] & \cdot \arrow[r, ""]
& \cdot
\end{tikzcd}
\end{equation*}

\textbf{Full rectangle pullback (black):}  For any $h_1$ and $h_2$, there is a unique $\overline{h}$ such that the black diagram commutes.

Since the right hand square is a pullback, for any $g_1$, there is a unique $\overline{g}$ such that $g_1 = p_{2\cdot}\overline{g}$. Since the rectangle is a pullback, for any $h_1$, there exists a unique $\overline{h}$ such that $p_{2\cdot}p_{12}\overline{h} = h_1$, and $p_{1\cdot}\overline{h} = h_2$. Set $g_1=h_1$, then $p_{2\cdot}p_{12}\overline{h} = p_{2\cdot}\overline{g}$, so $p_{12}\overline{h} = \overline{g}$. $\overline{g}$ can be regarded as an arbitrary $f_1$, as there is a one to one correspondence with $\overline{g}$ and the arbitrary choice of $g_1$, or equivalently, $h_1$.

\subsubsection*{5.1.36}

\paragraph{(a)}

If $(L \xrightarrow{p_I} D(I))_{i \in I}$ is a limit cone, there exists a unique $h$ such that $p_I \circ h = f_I$. However we are given that $p_I \circ h = p_I \circ h^\prime = f_I$, so $h$ must equal $h^\prime$.

\paragraph{(b)}

When $I$ is the two object discrete category, say $X \times Y$, $\mathcal{A}=\mathbf{Set}$, and $A=1$, the statement in (a) says if $x = x^\prime, y = y^\prime$, then $(x, y) = (x^\prime, y^\prime)$.

\subsubsection*{5.1.37}

For any $A \in \mathcal{A}$, and all maps $I \xrightarrow{u} J$, a cone on D is

\begin{equation}
\label{eqn:5137}
\begin{tikzcd}
A \arrow{r}{f_I} \arrow{rd}{f_J}
  & D(I) \arrow{d}{Du} \\
    & D(J)
  \end{tikzcd}
\end{equation}

A limit of $D$ is a cone $(L \xrightarrow{p_I} D(I))_{I \in \mathbf{I}}$, such that for any cone on $D$ with vertex A (\ref{eqn:5137}), there exists a unique map $\overline{f}\colon A \rightarrow L$ such that $p_J \circ \overline{f} = f_J$, for all $J \in \mathbf{I}$.

We have the set $\{ (x_I)_{I \in \mathbf{I}} | x_I \in D(I) \text{ for all } I\in \mathbf{I} \text{ and } (Du)(x_I)=x_J, \text{ for all } I \xrightarrow{u} J \text{ in } \mathbf{I} \}.$ The product limit formed is easier seen graphically. There is a family of maps for each $I \in \mathbf{I}$, each with

\begin{equation*}
\begin{tikzcd}
1 \arrow{r}{f_I} \arrow{rd}{f_J}
  & x_I \in D(I) \arrow{d}{Du} \\
    & x_J \in D(J)
  \end{tikzcd}
\end{equation*}


Then fix $p_J = Du$, $\overline{f}=f_I$,  and we have from the definition of a cone and (\ref{eqn:5137}) above $p_J \circ \overline{f} = f_J$, for all $J \in \mathbf{I}$. $\overline{f}$ is also unique. To see this assume there are two maps $\overline{f}$ and $\overline{f}^\prime$, that make the above triangle commute. Then $Du \circ \overline{f} = Du \circ  \overline{f}^\prime$, for all maps $I \rightarrow J$. Set $I=J$ to retrieve $\overline{f} = \overline{f}^\prime$. This family of maps we have described is precisely the definition of a product given in 5.1.7.  So the set of $x_I$ can be written $\prod_{I \in \mathbf{I}}D(I)$.

So if any cone exists in $\mathbf{Set}$, then a limit exists. Does a cone always exist in $\mathbf{Set}$?

\subsubsection*{5.1.38}
\paragraph{(a)}

We are given maps $s$ and $t$,

\begin{equation*}
  \prod_{I \in \mathbf{I}} D(I) \mathop{\rightrightarrows}_t^{s} \prod_{J \xrightarrow{u} K \text{ in } \mathbf{I}} D(K)
\end{equation*}

The $u$-component of $s$ is the composite
\begin{equation*}
  \prod_{I \in \mathbf{I}}D(I) \xrightarrow{pr_J} D(J) \xrightarrow{Du} D(K)
\end{equation*}

The $u$-component of $t$ is $pr_K$.

The fork property of the equalizer, says that the below diagram commutes for all maps $u$, $J \xrightarrow{u} K \text{ in } \mathbf{I}$, essentially that maps $(A \rightarrow D(J))_{J \in \mathbf{I}}$ are a cone on $D$.

\begin{equation}
\label{eqn:5138a}
\begin{tikzcd}
  A \arrow{d} &\\
  \prod_{I \in \mathbf{I}}D(I) \arrow{r}{pr_J} \arrow{rd}{pr_K}
  & D(J) \arrow{d}{Du} \\
    & D(K)
  \end{tikzcd}
\end{equation}

The other important property of the equalizer is that for any fork, or as above, cone, there exists a unique map $\overline{f} \colon A \rightarrow L$ such that

\begin{equation}
\label{eqn:5138b}
\begin{tikzcd}
  A \arrow{d}{\overline{f}} \arrow{dr}{f} & \\
  L \arrow{r}{i} & \prod_{I \in \mathbf{I}}D(I)
\end{tikzcd}
\end{equation}

commutes.

Now $(L \xrightarrow{pr_J \circ i} D(J))_{J \in \mathbf{J}}$ is a cone, as it factors through $\prod_{I \in \mathbf{I}}D(I)$, as $A$ does in (\ref{eqn:5138a}). (\ref{eqn:5138b}) also implies $pr_J \circ i \circ \overline{f} = f_J$ for all $J$, where $f_J \colon A \rightarrow D(J) = pr_J \circ f$.

\paragraph{(b)}

The definition of a finite limit is a limit of shape $\mathbf{I}$ for some finite category $I$. So to show a limit is finite, we must show the diagram the limit maps into is indexed by a finite category. Finite categories have only finitely many maps. So binary products, terminal objects, equalizers and pullbacks are all finite limits. From part (a) we know if $\mathcal{A}$ has all products and equalizers then $\mathcal{A}$ has all limits. If we however restrict the products to binary products, then by definition limits of $\mathcal{A}$ will be finite.

\subsubsection*{5.1.39}

A pullback (5.7) from page 114, with $Z$ as the terminal object, collapses to a binary product. The key point here is that the limit is unique up to isomorphism, so limits in a category with pullbacks and a terminal object are binary products, and hence finite.

\subsubsection*{5.1.40}

We are given $X \xrightarrow{m} A$, and $X^\prime \xrightarrow{m^\prime} A$ are monics in $\mathbf{Set}$. $\mathbf{Monic}(A)$ is the full subcategory of the slice category $\mathcal{A} / A$, whose objects have as their maps the monics. Recall in $\mathcal {A}/A$, objects are tuples $(X, m)$ such that the following diagram commutes
\begin{equation*}
\begin{tikzcd}
X \arrow{dr}{m} \arrow{rr}{f}
& & X^\prime \arrow{dl}{m^\prime} \\
& A
\end{tikzcd}
\end{equation*}

\paragraph{Isomorphic implies equal images:}
Note that if $m$ and $m^\prime$ are isomorphic, then $f$ must be a bijection. The bijection can then be written $m = m^\prime \circ f$, and $m^\prime = m \circ f^{-1}$. Note also the m$m$ and $m\prime$, by virtue of them as monics, are injective. \textbf{Intuition:} We can essentially roundtrip on the triangle above, starting from an element in the image of $m$ (or conversely $m^\prime$), and map it to an element in the image of $m^\prime$ (respectively $m$).  Explicitly we can write $\{ m(x), x \in X \} = \{ m^\prime \circ f (x), x \in X \} = \{ m^\prime (x), x \in X^\prime \}$.

\paragraph{Equal images implies isomorphic:}

If images of $m$ and $m^\prime$ are equal,

$|{m^{\prime}}^{-1}(A)| = |m^{-1}(A)|$, which implies a bijection between $X$ and $X^\prime$, and hence a bijection between maps $m$ and $m^\prime$ as in the previous paragraph.

\subsubsection*{5.1.41}

\begin{equation*}
\begin{tikzcd}
Y
\arrow[drr, bend left, "p"]
\arrow[ddr, bend right, "q"]
\arrow[dr, dashed, "\overline{f}" description] & \\
& X \arrow[r, "1"] \arrow[d, "1"]
& X \arrow[d,"f"] \\
& X \arrow[r, "f"]
& Y
\end{tikzcd}
\end{equation*}

From the pullback diagram, for all commuting maps, that is for all
$p, q, f \circ p = f \circ q \implies p = q$, if and only if the diagram above is a pullback.

\subsubsection*{5.1.42}

The given square is a pullback, which means for a fixed $f,m,f^\prime,m^\prime$, any other commuting square factors through it as follows.

\begin{equation*}
\begin{tikzcd}
Y
\arrow[drr, bend left, "f \overline{f}"]
\arrow[ddr, bend right, "m^\prime\overline{f}"]
\arrow[dr, dashed, "\overline{f}" description] & \\
& X^\prime \arrow[r, "f"] \arrow[d, "m^\prime"]
& X \arrow[d,"m"] \\
& A^\prime \arrow[r, "f^\prime"]
& A
\end{tikzcd}
\end{equation*}

We know from the properties of a pullback that $\overline{f}: Y \rightarrow X^\prime$ is unique for each distinct pair of maps, $Y \rightarrow X$, and $Y \rightarrow A^\prime$, such that the diagram above commutes.

In the following we use the contrapositive form of monic, so for maps $x, x^\prime$, $f$ is monic if $x \neq x^\prime \implies f \circ x \neq f \circ x^\prime$.

Now we know $m$ is monic, so consider two distinct $\overline{f}_1$ and $\overline{f}_2$ in respect of two commuting diagrams as above. There must indeed be two distinct $mf\overline{f}_1 \neq mf\overline{f}_2$, such that each respective diagram commutes. Since the outer arrows commute, $mf\overline{f}_1 = f^\prime m^\prime \overline{f}_1$, and $mf\overline{f}_2 = f^\prime m^\prime \overline{f}_2$. So $f^\prime m^\prime \overline{f}_1 \neq f^\prime m^\prime \overline{f}_2 \implies m^\prime \overline{f}_1 \neq m^\prime \overline{f}_2$, and $m^\prime$ is monic.

\subsubsection*{5.1.42}

The equaliser is a map $f$ below such that $si = ti$, together with a universal property. The coequaliser is a map $p$ satisfying $ps=pt$, and universal with this property.

\begin{equation*}
\begin{tikzcd}
E  \ar[r,"i"] &
X \ar[r,shift left=.75ex,"s"]
  \ar[r,shift right=.75ex,swap,"t"] & Y
  \ar[r,"p"] & C
\end{tikzcd}
\end{equation*}

If $f$ is isomorphic then there is a $\overline{f}$ such that $f\overline{f}=1_E, \overline{f}f=1_X$, so $sf\overline{f} = s = tf\overline{f}=t$.

In the opposite direction, we need to show if $s=t$, then the equaliser exists and is isomorphic. To do this we will use the universal property of the equaliser. Specifically, any $f$ that is a fork factors through $i$ as below

\begin{equation*}
\begin{tikzcd}
  E \ar[r,"i"] & X \ar[r,shift left=.75ex,"s"]
  \ar[r,shift right=.75ex,swap,"t"] & Y \\
  A \ar[u,"\overline{f}"] \ar[ur,swap,"f"] &
\end{tikzcd}
\end{equation*}

Since $s=t$ we can choose any function $f$ and it will be a fork, and hence an equaliser exists. Immediately we can see that if we choose $f=1_X$ then we have $i \circ \overline{i}=1_X$, where $\overline{i}$ is the unique morphism depicted by $\overline{f}$ in the diagram below. Now we need to show $\overline{i} \circ i = 1_E$. Put $f=i\overline{i}i$ below, then there is a unique $h$ such that

\begin{equation}
  \label{5142eqn1}
  i\overline{i}i=ih.
\end{equation}

This implies $h=\overline{i}i$. Substituting $i\overline{i}=1_X$ into (\ref{5142eqn1}) yields $h=1_E$. So $\overline{i}i = 1_E$.

Proof works the same dualised for the coequaliser.
\end{document}
