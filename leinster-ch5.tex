\documentclass{article}
\usepackage{amsmath}
\usepackage{amssymb}
\usepackage{dsfont}
\usepackage{tikz-cd}
\usetikzlibrary{matrix}
\usepackage{lipsum}
\usepackage{titlesec}
\usepackage{parskip}
\begin{document}

\title{Leinster - Basic Category Theory - Selected problem solutions for Chapter 5}
\author{Adam Barber}

\maketitle

\subsubsection*{5.1.34}

The equaliser square is not necessarily a pullback. There is no reason why any function into the $X$ would commute with a unique function into $E$, composed with $i$.

The converse is true though, a pullback implies an equaliser, when the square is set up as in the question.

\subsubsection*{5.1.35}

Suppose the right hand square is a pullback. We need to prove the left hand square is a pullback if and only if the full rectangle, which composes both squares, is a pullback.

\paragraph{Only if}

Assume the left hand square and right hand squares are pullbacks. Show full rectangle is a pullback, that is show $g_1 = p_{2\cdot}p_{12}\overline{f}$, and $f_2 = p_{1\cdot}\overline{f}$

\begin{equation*}
\begin{tikzcd}
A
\arrow[drr, bend left, "f_1"]
\arrow[drr, "\overline{g}", dashed, yshift=0.2ex, red]
\arrow[ddr, bend right, "f_2"]
\arrow[dr, dashed, "\overline{f}" description]
\arrow[drrr, bend left=50, dotted, red, yshift=0.3ex, "g_1"]
\arrow[ddrr, bend right=90, dotted, red, yshift=-0.8ex, "g_2"'] & & \\
& P_1 \arrow[r, "p_{12}"] \arrow[d, "p_{1\cdot}"] & P_2 \arrow[r, "p_{2\cdot}", red] \arrow[d, "", red]
& \cdot \arrow[d, "" , red] \\
& \cdot \arrow[r, ""] & \cdot \arrow[r, "", red]
& \cdot
\end{tikzcd}
\end{equation*}

\textbf{Left square pullback (black):} For any $f_1$ and $f_2$, there is a unique map $\overline{f}$ such that the left square above commutes.

\textbf{Right square pullback (red):}  For any $g_1$ and $g_2$, there is a unique $\overline{g}$ such that the red diagram commutes.

Due to the left hand square being a pullback, for each $f_1$, and $f_2$, there is a unique map $\overline{f}$ such that $f_2 = p_{1\cdot}\overline{f}$ and $f_1 = p_{12}\overline{f}$. Set $f_1 = \overline{g}$.   From the right hand side being a pullback,  $g_1 = p_{2\cdot}\overline{g}=p_{2\cdot}p_{12}\overline{f}$ as required.

\paragraph{If}

Assume the outer rectangle and right hand square are both pullbacks. Show the left hand side square is a pullback, that is $f_1 = p_{12}\overline{h}$, and $h_2=p_{1\cdot}\overline{h}$, for any $f_1,  h_2$.

\begin{equation*}
\begin{tikzcd}
A
\arrow[drrr, bend left, "h_1"]
\arrow[drr, "\overline{g}", yshift=0.2ex, dotted, red]
\arrow[ddr, bend right, "h_2"]
\arrow[dr, dotted, "\overline{h}" description]
\arrow[drrr, bend left=50, dotted, red, yshift=0.5ex, "g_1"]
\arrow[ddrr, bend right=90, dotted, red, yshift=-0.8ex, "g_2"'] & & \\
& P_1 \arrow[r, "p_{12}"] \arrow[d, "p{1\cdot}"] & P_2 \arrow[r, "p_{2\cdot}"] \arrow[d, ""]
& \cdot \arrow[d, ""] \\
& \cdot \arrow[r, ""] & \cdot \arrow[r, ""]
& \cdot
\end{tikzcd}
\end{equation*}

\textbf{Full rectangle pullback (black):}  For any $h_1$ and $h_2$, there is a unique $\overline{h}$ such that the black diagram commutes.

Since the right hand square is a pullback, for any $g_1$, there is a unique $\overline{g}$ such that $g_1 = p_{2\cdot}\overline{g}$. Since the rectangle is a pullback, for any $h_1$, there exists a unique $\overline{h}$ such that $p_{2\cdot}p_{12}\overline{h} = h_1$, and $p_{1\cdot}\overline{h} = h_2$. Set $g_1=h_1$, then $p_{2\cdot}p_{12}\overline{h} = p_{2\cdot}\overline{g}$, so $p_{12}\overline{h} = \overline{g}$. $\overline{g}$ can be regarded as an arbitrary $f_1$, as there is a one to one correspondence with $\overline{g}$ and the arbitrary choice of $g_1$, or equivalently, $h_1$.

\subsubsection*{5.1.36}

\paragraph{(a)}

If $(L \xrightarrow{p_I} D(I))_{i \in I}$ is a limit cone, there exists a unique $h$ such that $p_I \circ h = f_I$. However we are given that $p_I \circ h = p_I \circ h^\prime = f_I$, so $h$ must equal $h^\prime$.

\paragraph{(b)}

When $I$ is the two object discrete category, say $X \times Y$, $\mathcal{A}=\mathbf{Set}$, and $A=1$, the statement in (a) says if $x = x^\prime, y = y^\prime$, then $(x, y) = (x^\prime, y^\prime)$.

\subsubsection*{5.1.37}

For any $A \in \mathcal{A}$, and all maps $I \xrightarrow{u} J$, a cone on D is

\begin{equation}
\label{eqn:5137}
\begin{tikzcd}
A \arrow{r}{f_I} \arrow{rd}{f_J}
  & D(I) \arrow{d}{Du} \\
    & D(J)
  \end{tikzcd}
\end{equation}

A limit of $D$ is a cone $(L \xrightarrow{p_I} D(I))_{I \in \mathbf{I}}$, such that for any cone on $D$ with vertex A (\ref{eqn:5137}), there exists a unique map $\overline{f}\colon A \rightarrow L$ such that $p_J \circ \overline{f} = f_J$, for all $J \in \mathbf{I}$.

We have the set $\{ (x_I)_{I \in \mathbf{I}} | x_I \in D(I) \text{ for all } I\in \mathbf{I} \text{ and } (Du)(x_I)=x_J, \text{ for all } I \xrightarrow{u} J \text{ in } \mathbf{I} \}.$ The product limit formed is easier seen graphically. There is a family of maps for each $I \in \mathbf{I}$, each with

\begin{equation*}
\begin{tikzcd}
1 \arrow{r}{f_I} \arrow{rd}{f_J}
  & x_I \in D(I) \arrow{d}{Du} \\
    & x_J \in D(J)
  \end{tikzcd}
\end{equation*}


Then fix $p_J = Du$, $\overline{f}=f_I$,  and we have from the definition of a cone and (\ref{eqn:5137}) above $p_J \circ \overline{f} = f_J$, for all $J \in \mathbf{I}$. $\overline{f}$ is also unique. To see this assume there are two maps $\overline{f}$ and $\overline{f}^\prime$, that make the above triangle commute. Then $Du \circ \overline{f} = Du \circ  \overline{f}^\prime$, for all maps $I \rightarrow J$. Set $I=J$ to retrieve $\overline{f} = \overline{f}^\prime$. This family of maps we have described is precisely the definition of a product given in 5.1.7.  So the set of $x_I$ can be written $\prod_{I \in \mathbf{I}}D(I)$.

So if any cone exists in $\mathbf{Set}$, then a limit exists. Does a cone always exist in $\mathbf{Set}$?

\subsubsection*{5.1.38}
\paragraph{(a)}

We are given maps $s$ and $t$,

\begin{equation*}
  \prod_{I \in \mathbf{I}} D(I) \mathop{\rightrightarrows}_t^{s} \prod_{J \xrightarrow{u} K \text{ in } \mathbf{I}} D(K)
\end{equation*}

The $u$-component of $s$ is the composite
\begin{equation*}
  \prod_{I \in \mathbf{I}}D(I) \xrightarrow{pr_J} D(J) \xrightarrow{Du} D(K)
\end{equation*}

The $u$-component of $t$ is $pr_K$.

The fork property of the equalizer, says that the below diagram commutes for all maps $u$, $J \xrightarrow{u} K \text{ in } \mathbf{I}$, essentially that maps $(A \rightarrow D(J))_{J \in \mathbf{I}}$ are a cone on $D$.

\begin{equation}
\label{eqn:5138a}
\begin{tikzcd}
  A \arrow{d} &\\
  \prod_{I \in \mathbf{I}}D(I) \arrow{r}{pr_J} \arrow{rd}{pr_K}
  & D(J) \arrow{d}{Du} \\
    & D(K)
  \end{tikzcd}
\end{equation}

The other important property of the equalizer is that for any fork, or as above, cone, there exists a unique map $\overline{f} \colon A \rightarrow L$ such that

\begin{equation}
\label{eqn:5138b}
\begin{tikzcd}
  A \arrow{d}{\overline{f}} \arrow{dr}{f} & \\
  L \arrow{r}{i} & \prod_{I \in \mathbf{I}}D(I)
\end{tikzcd}
\end{equation}

commutes.

Now $(L \xrightarrow{pr_J \circ i} D(J))_{J \in \mathbf{J}}$ is a cone, as it factors through $\prod_{I \in \mathbf{I}}D(I)$, as $A$ does in (\ref{eqn:5138a}). (\ref{eqn:5138b}) also implies $pr_J \circ i \circ \overline{f} = f_J$ for all $J$, where $f_J \colon A \rightarrow D(J) = pr_J \circ f$.

\paragraph{(b)}

The definition of a finite limit is a limit of shape $\mathbf{I}$ for some finite category $I$. So to show a limit is finite, we must show the diagram the limit maps into is indexed by a finite category. Finite categories have only finitely many maps. So binary products, terminal objects, equalizers and pullbacks are all finite limits. From part (a) we know if $\mathcal{A}$ has all products and equalizers then $\mathcal{A}$ has all limits. If we however restrict the products to binary products, then by definition limits of $\mathcal{A}$ will be finite.

\subsubsection*{5.1.39}

A pullback (5.7) from page 114, with $Z$ as the terminal object, collapses to a binary product. The key point here is that the limit is unique up to isomorphism, so limits in a category with pullbacks and a terminal object are binary products, and hence finite.

\subsubsection*{5.1.40}

We are given $X \xrightarrow{m} A$, and $X^\prime \xrightarrow{m^\prime} A$ are monics in $\mathbf{Set}$. $\mathbf{Monic}(A)$ is the full subcategory of the slice category $\mathcal{A} / A$, whose objects have as their maps the monics. Recall in $\mathcal {A}/A$, objects are tuples $(X, m)$ such that the following diagram commutes
\begin{equation*}
\begin{tikzcd}
X \arrow{dr}{m} \arrow{rr}{f}
& & X^\prime \arrow{dl}{m^\prime} \\
& A
\end{tikzcd}
\end{equation*}

\paragraph{Isomorphic implies equal images:}
Note that if $m$ and $m^\prime$ are isomorphic, then $f$ must be a bijection. The bijection can then be written $m = m^\prime \circ f$, and $m^\prime = m \circ f^{-1}$. Note also the m$m$ and $m\prime$, by virtue of them as monics, are injective. \textbf{Intuition:} We can essentially roundtrip on the triangle above, starting from an element in the image of $m$ (or conversely $m^\prime$), and map it to an element in the image of $m^\prime$ (respectively $m$).  Explicitly we can write $\{ m(x), x \in X \} = \{ m^\prime \circ f (x), x \in X \} = \{ m^\prime (x), x \in X^\prime \}$.

\paragraph{Equal images implies isomorphic:}

If images of $m$ and $m^\prime$ are equal,

$|{m^{\prime}}^{-1}(A)| = |m^{-1}(A)|$, which implies a bijection between $X$ and $X^\prime$, and hence a bijection between maps $m$ and $m^\prime$ as in the previous paragraph.

\subsubsection*{5.1.41}

\begin{equation*}
\begin{tikzcd}
Y
\arrow[drr, bend left, "p"]
\arrow[ddr, bend right, "q"]
\arrow[dr, dashed, "\overline{f}" description] & \\
& X \arrow[r, "1"] \arrow[d, "1"]
& X \arrow[d,"f"] \\
& X \arrow[r, "f"]
& Y
\end{tikzcd}
\end{equation*}

From the pullback diagram, for all commuting maps, that is for all
$p, q, f \circ p = f \circ q \implies p = q$, if and only if the diagram above is a pullback.

\subsubsection*{5.1.42}

The given square is a pullback, which means for a fixed $f,m,f^\prime,m^\prime$, any other commuting square factors through it as follows.

\begin{equation*}
\begin{tikzcd}
Y
\arrow[drr, bend left, "f \overline{f}"]
\arrow[ddr, bend right, "m^\prime\overline{f}"]
\arrow[dr, dashed, "\overline{f}" description] & \\
& X^\prime \arrow[r, "f"] \arrow[d, "m^\prime"]
& X \arrow[d,"m"] \\
& A^\prime \arrow[r, "f^\prime"]
& A
\end{tikzcd}
\end{equation*}

We know from the properties of a pullback that $\overline{f}: Y \rightarrow X^\prime$ is unique for each distinct pair of maps, $Y \rightarrow X$, and $Y \rightarrow A^\prime$, such that the diagram above commutes.

In the following we use the contrapositive form of monic, so for maps $x, x^\prime$, $f$ is monic if $x \neq x^\prime \implies f \circ x \neq f \circ x^\prime$.

Now we know $m$ is monic, so consider two distinct $\overline{f}_1$ and $\overline{f}_2$ in respect of two commuting diagrams as above. There must indeed be two distinct $mf\overline{f}_1 \neq mf\overline{f}_2$, such that each respective diagram commutes. Since the outer arrows commute, $mf\overline{f}_1 = f^\prime m^\prime \overline{f}_1$, and $mf\overline{f}_2 = f^\prime m^\prime \overline{f}_2$. So $f^\prime m^\prime \overline{f}_1 \neq f^\prime m^\prime \overline{f}_2 \implies m^\prime \overline{f}_1 \neq m^\prime \overline{f}_2$, and $m^\prime$ is monic.

\subsubsection*{5.2.21}

The equaliser is a map $f$ below such that $si = ti$, together with a universal property. The coequaliser is a map $p$ satisfying $ps=pt$, and universal with this property.

\begin{equation*}
\begin{tikzcd}
E  \ar[r,"i"] &
X \ar[r,shift left=.75ex,"s"]
  \ar[r,shift right=.75ex,swap,"t"] & Y
  \ar[r,"p"] & C
\end{tikzcd}
\end{equation*}

If $f$ is isomorphic then there is a $\overline{f}$ such that $f\overline{f}=1_E, \overline{f}f=1_X$, so $sf\overline{f} = s = tf\overline{f}=t$.

In the opposite direction, we need to show if $s=t$, then the equaliser exists and is isomorphic. To do this we will use the universal property of the equaliser. Specifically, any $f$ that is a fork factors through $i$ as below

\begin{equation*}
\begin{tikzcd}
  E \ar[r,"i"] & X \ar[r,shift left=.75ex,"s"]
  \ar[r,shift right=.75ex,swap,"t"] & Y \\
  A \ar[u,"\overline{f}"] \ar[ur,swap,"f"] &
\end{tikzcd}
\end{equation*}

Since $s=t$ we can choose any function $f$ and it will be a fork, and hence an equaliser exists. Immediately we can see that if we choose $f=1_X$ then we have $i \circ \overline{i}=1_X$, where $\overline{i}$ is the unique morphism depicted by $\overline{f}$ in the diagram below. Now we need to show $\overline{i} \circ i = 1_E$. Put $f=i\overline{i}i$ below, then there is a unique $h$ such that

\begin{equation}
  \label{5142eqn1}
  i\overline{i}i=ih.
\end{equation}

This implies $h=\overline{i}i$. Substituting $i\overline{i}=1_X$ into (\ref{5142eqn1}) yields $h=1_E$. So $\overline{i}i = 1_E$.

Proof for the coequaliser works the same, but dualised.

\subsubsection*{5.2.22}

\paragraph{(a)}

The coequaliser of

\begin{equation}
\label{eqn5143}
\begin{tikzcd}
  X \ar[r,shift left=.75ex,"f"]
    \ar[r,shift right=.75ex,swap,"1"] & X
\end{tikzcd}
\end{equation}

in \textbf{Set} is described as follows. Let $\sim$ be the equivalence relation between domain and codomain of $f$, $x \sim fx$, for all $x \in X$. The coequaliser (\ref{eqn5143}) is then the quotient map $p\colon X \rightarrow X/\sim$.

\paragraph{(b)}

The quotient needs to be something quite degenerate in that no subsets of it can be open, in order to form an indiscrete topology as the question requires. So this rules out any type of $g$-orbit quotient which would take the points on the circle to $g$-orbits on the plane, e.g $e^{i2\pi z} \mapsto z \mod 1$. Any such mapping would map an open arc in $S^1$ to a family of open sets in $\mathds{R}$, and hence not have the indiscrete topology.

So here is a solution. Let $X$ be the circle $S^1= \{e^{i2\pi x}, x \in [0,1)\}$. Define an equivalence relation $x \sim y$ if $y - x \in \mathds{Q}$. A set $V=S^1/\mathds{Q}$ is open if $\{x \in [0,1)\colon [x] \in V \}$ is open in $[0,1)$. Here $[x]$ is the equivalence class of $x$, i.e
\begin{equation*}
  \{ x \in [0,1)\colon y - x \in \mathds{Q} \}
\end{equation*}

If $V \subset S^1/\mathds{Q}$ is open and nonempty then $\{ x \in [0,1): [x] \in V  \}$ should contain an interval $(a,b), a \ge 0, a < 1$. So $(a,b) \subset \{ x \in [0,1): [x] \in V  \}$, which means that
\begin{equation*}
  x \in (a,b) \implies [x] \in V
\end{equation*}

But consider $z \in [0,1)$. Then there exists a $z^\prime \in (a,b)$, such that $z - z^\prime \in \mathds{Q}$.

Hence $[z] = [z^\prime] \in V$. Hence $V = S^1/\mathds{Q}$. So our quotient has the indiscrete topology.  \footnote{https://math.stackexchange.com/questions/4452608/quotient-map-on-s1-such-that-that-the-quotient-is-an-uncountable-space-with-the}
\paragraph{Cardinality}  Informally the elements of the quotient are of the form $\{ a + q, q \in \mathds{Q} \}$, where $a \in [0,1)$, so uncountable as is $[0,1)$.

\subsubsection*{5.2.23}
\paragraph{(a)}

We have the inclusion $f: (\mathds{N},+,0) \rightarrow (\mathds{Z},+,0)$. If $g\circ f = g^\prime\circ f$, we need to show $g=g^\prime$. $g\circ f$ is essentially a group homomorphism restricted to a domain of $\mathds{N}$, whereas $g$ is the respective homomorphism on the expanded domain of $\mathds{Z}$. The idea here is to stitch together a group homomorphism on the expanded domain using the group homomorphism on $\mathds{N}$.

Define, for $x \in \mathds{N}$
\begin{align*}
  h(x) &= \begin{cases}
    g \circ f(x), &  x >=0,  \\
    g \circ f(-x), &  x < 0
  \end{cases} \\
\end{align*} and define $h^\prime$ analogously.

Since $g \circ f(-x) = -g \circ f (x) = -g^\prime \circ f (x) = g^\prime \circ f (-x)$, then $h=h^\prime$ on $\{\forall z \in \mathds{Z}\}$.

\paragraph{(b)}
We have $i: \mathds{Z} \rightarrow \mathds{Q}$, and $hi=h^\prime i$. We need to show $h=h^
\prime$, on the full domain of $\mathds{Q}$.
A rational number is defined as $p/q, p,q \in \mathds{Z}, q \neq 0$. The answer is a similar concept to (a), stitch together a ring homomorphism from the $hi$ on $\mathds{Z}$, and show equality holds on the full domain of $\mathds{Q}$.

Define $h(x) = h(p)h(1/q) = \frac{hi(p)}{hi(q)}$, and it follows that $h=h^\prime$ as required.

\subsubsection*{5.2.24}
\paragraph{(a)}

\begin{equation*}
\begin{tikzcd}
X \arrow{rr}{f} & & X^\prime \\
& \arrow{ul}{e} A \arrow{ur}{e^\prime}
\end{tikzcd}
\end{equation*}

$e, e^\prime$ in $\mathbf{Epic}(A)$. The equivalence relation on $A$ is written

\begin{equation*}
a \sim a^\prime \implies f(a) = f(a^\prime)
\end{equation*}

\paragraph{Same equivalence relation implies isomorphism}

Consider the following two equivalence relations
\begin{equation}
  \label{eqn5224a1}
  a \sim a^\prime \implies e(a) = e(a^\prime),
\end{equation}
\begin{equation}
  \label{eqn5224a2}
  a \approx a^\prime  \implies e^\prime(a) = e^\prime(a^\prime)
\end{equation}
From Chapter 3 p70, we know that a map $p: A \rightarrow A/\sim$ sending an element to its equivalence class, has a universal property. Any function $f: A \rightarrow B$ such that
\begin{align*}
  \forall a, a^\prime \in A, a \sim a^\prime \implies f(a)=f(a^\prime)
\end{align*}

factorises uniquely through $p$, as in the diagram
\begin{equation*}
\begin{tikzcd}
A \arrow[rd,"f"] \arrow[r, "p"] & A/\sim \arrow[d, "\overline{f}"] \\
& C
\end{tikzcd}
\end{equation*}

Now if indeed both $e$ and $e^\prime$ induce the \textbf{same} equivalance relation, then $a \sim a^\prime$ also implies $e^\prime(a) = e^\prime(a^\prime)$. We can then use the universal property of the relation $\sim$ as follows

\begin{equation*}
\begin{tikzcd}
X \arrow[rd,"e^\prime"] \arrow[r, "e"] & X/\sim \arrow[d, "\overline{e}"] \\
& A^\prime
\end{tikzcd}
\end{equation*}

And similarly, with the relation $\approx$ (\ref{eqn5224a2}) we have

\begin{equation*}
\begin{tikzcd}
X \arrow[rd,"e"] \arrow[r, "e^\prime"] & X/\approx \arrow[d, "\overline{e^\prime}"] \\
& A
\end{tikzcd}
\end{equation*}

Equating from the diagrams, $A=X/\sim$, and $A^\prime=X/\approx$. So the quotient sets are isomorphic, and isomorphisms of $e$ and $e^\prime$ follow.

\textbf{Note} The surjective property of $e$ and $e^\prime$ is implicit in them being quotient maps. Speicifcally each equivalence class has at least one member.

\textbf{Aside} More generally isomorphisms are coequalizers of identity morphisms:

\begin{equation*}
\label{eqn5143}
\begin{tikzcd}
  X \ar[r,shift left=.75ex,"1"]
    \ar[r,shift right=.75ex,swap,"1"] & X
\end{tikzcd}
\end{equation*}

\paragraph{Isomorphism implies same equivalence relation}

If $e$ and $e^\prime$ are isomorphic then there is an injective $f$ such that

\begin{align*}
  e^\prime &= f e, \;\text{and}\\
  e &= f^{-1}e^\prime
\end{align*}

So any relation $R = \{(a,a^\prime)\colon e(a) = e(a^\prime)\} = \{(a,a^\prime)\colon e^\prime(a) = e^\prime(a^\prime)\}$ by injectivity of $f$.

\subsubsection*{5.2.25}
\paragraph{(a)}

An \textbf{equaliser} of $s$ and $t$, is a map $i$ such that $s$ is a fork, that is $si=ti$. Furthermore any other fork $f$ factors through $i$ as below

\begin{equation*}
\begin{tikzcd}
A \arrow[rd,"f"] \arrow[d, "\overline{e}"] & \\
E \arrow[r,"i"] & X
\end{tikzcd}
\end{equation*}

\paragraph{Split monic $\implies$ regular monic}

If we have $em = 1_A$, then $m$ is an equaliser of

\begin{equation*}
\label{eqn5143}
\begin{tikzcd}
  X \ar[r,shift left=.75ex,"1"]
    \ar[r,shift right=.75ex,swap,"1"] & X
\end{tikzcd}
\end{equation*}

and the universal map is

\begin{equation*}
\begin{tikzcd}
A \arrow[rd,"f"] \arrow[d, "fe"] & \\
E \arrow[r,"m"] & B
\end{tikzcd}
\end{equation*}

\paragraph{Regular monic $\implies$ monic}

$m$ has to be monic for the universal property of $m$ as an equaliser to hold. The universal property states that for all forks $f$ factor through $m$. To see this, consider two such forks $f_1 \neq f_2$, with their corresponding unique maps $\overline{f}_1 \neq \overline{f}_2$. To get a contradiction assume $m$ is not monic and $m\overline{f}_1 = m\overline{f}_2$. Then the triangle does not commute and the universal property does not hold.

\textbf{Alternative more elegant proof\footnote{https://math.stackexchange.com/questions/81296/every-equalizer-is-monic}}

This is a much simpler proof than my attempt above, that an equaliser is monic.

Assume $i\colon E \rightarrow A$ equalises $f\colon A \rightarrow B, g\colon A \rightarrow B$.  For maps $j, l: D \rightarrow E$ such that $i \circ j = i \circ l$, we know $f (i \circ j) = (f \circ i) \circ j = (g \circ i) \circ j = g (i \circ j)$. Reminder: maps in a category are associative. So $i \circ j$ is a fork. By the universal property there exists a \textit{unique} $k$ such that $i \circ k = i \circ j$. So $j = k = l$, and $i \circ j = i \circ l \implies j = l$ as required.

\paragraph{(b)}

Need to find two maps that a monic $m\colon A \rightarrow B$ in \textbf{Ab} equalises. They are $g: B \rightarrow B/\text{im}(M)$, and the zero map $0: B \rightarrow 0$. The zero object in \textbf{Ab} is by definition the initial and terminal object, which is the trivial group. So $m$ equalises $f$ and $g$. You can see $g$ composing $m$ means the codomain will be $\text{im}(m)/\text{im}(m)$ which is isomorphic to $0$.

For the second part need to find a monic $m$ which we can't invert. $f:\mathds{Z} \rightarrow \mathds{Q}$ works.

\paragraph{(c)}

\begin{equation*}
\label{eqn5143}
\begin{tikzcd}
 A \ar[r, "m"] &
 B \ar[r,shift left=.75ex,"f"]
   \ar[r,shift right=.75ex,swap,"g"] & C
\end{tikzcd}
\end{equation*}

The question is to find a monic in $\mathbf{Top}$ which is not regular.

Some background, and my progress so far:

The monics $m: A \rightarrow B$ in $\mathbf{Top}$ are injections. So if we consider the restriction of $B$ to the image of $m$, and equip it with the subspace topology, then we have a homeomorphism from $A$ onto $m[A]$, or an embedding of $A$.   Now consider two maps $f,g: B \rightarrow \{0,1\}$, where {0,1} has the indiscrete topology, $f$ maps everything to $1$ and $g$ maps the elements of $m[A]$ to $1$. Then the equalizer $fm=gm$ represents the subspace embeddings. Specifically, the equalizer is a set consisting of all embeddings of $X$ for all $X \subseteq A$. The \textbf{regular} monics are those where an equalizer exists, and where an equalizer exists in $\mathbf{Top}$, it is the subspace embeddings as described. This is because equalizers are a limit, and limits are unique up to isomorphism.

Now we need a monic that is \textbf{not} regular. This is a monic that does not admit an equalizer of some $f, g$, as above. We know all regular monics in \textbf{Top} are the inclusion of a subspace. We also know that all injections are monics. So we simply need to find an injection between spaces which is not the inclusion of a subspace.
So equip $A$ with a finer topology than that induced by the subspace topology of $B$. Under a finer topology for $A$ there is at least one open set in $A$ mapping to a closed set in $m(A)$. However this means we have a $g^{-1}(b), b \in {0, 1}$ that is closed, hence $g$ is not continuous at this point, so no such pair of mappings in \textbf{Top} exist.

\end{document}
