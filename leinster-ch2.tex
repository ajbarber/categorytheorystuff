\documentclass{article}
\usepackage{amsmath}
\usepackage{dsfont}
\usepackage{tikz-cd}
\usetikzlibrary{matrix}
\usepackage{lipsum}
\usepackage{titlesec}
\usepackage{parskip}
\begin{document}

\title{Leinster - Basic Category Theory - Selected problem solutions for Chapter 2}
\author{Adam Barber}

\maketitle
\subsubsection*{2.1.16}
\paragraph{(a)}

Interesting adjoint functors to $G$-sets.

The trivial group functor $I$ sends a set to a $\mathbf{G}$-set with the trivial action $gx = x$. Interesting functors

Orbit functor sends a $G$-set with underlying set elements $a$ of $A$ to:

$A_G = \{ g \cdot a, g \in G \}$

Fixed point functor sends a $G$-set  with underlying set elements $a$ of $A$ to:

$A^G = \{ a \text{ such that } g \cdot a = a \text{ for all } g \in G, a \in A \}$

\paragraph{Fixed point functor - right adjoint}

Morphisms in a $G$-set are functions on the underlying set, where $f$ commutes with $g$ for every $g \in G$.

There is a bijection for each $A \in \mathbf{Set}$ and $B \in [G, \mathbf{Set}]$ as follows

\begin{align*}
  [G, \mathbf{Set}](I(A), B) &\rightarrow \mathbf{Set}(A, B^G) \\
  \psi &\mapsto \overline{\psi}
\end{align*}

$\overline{\psi}$ sends each element $a$ of $A$ to $\psi(a)$ if $g \cdot a = a$, otherwise it sends $a$ to $\psi(\emptyset)$.

\begin{align*}
  \mathbf{Set}(A, B^G) &\rightarrow [G, \mathbf{Set}](I(A), B) \\
  \phi &\mapsto \overline{\phi}
\end{align*}

$\phi$ sends each $a \in A$ in the underlying set of the $G$-set to the $G$-set $(g, \overline{\phi}(a)), g \in G$.

\paragraph{Orbit functor - left adjoint}

There is a bijection for each $A \in [G, \mathbf{Set}]$ and $B \in \mathbf{Set}$ as follows

\begin{align*}
  \mathbf{Set}(A_G, B) &\rightarrow [G, \mathbf{Set}](A, I(B)) \\
  \psi &\mapsto \overline{\psi}
\end{align*}

So each morphism in $\mathbf{Set}$ sends the set formed by the orbits of an element $a$ of $A$, call this $a_G$, to $\psi(a_G)$, where $\psi$ is a function of sets.
Choose a $G$-set morphism $\overline{\psi} = \psi$, where $\overline{\psi}$ commutes with $g$ for every $g$ in $G$.

\begin{align*}
  [G, \mathbf{Set}](A, I(B)) &\rightarrow \mathbf{Set}(A_G, B)\\
  \phi &\mapsto \overline{\phi}
\end{align*}

Choose $\overline{\phi}$ to be a disjoint union of each orbit of $a$ in $A$, $\overline{\phi}(a)$ =  $\amalg\{\phi(g \cdot a), g \in G\}$

\subsubsection*{2.1.17}

Write $\mathcal{O}(X)$ for the poset of open subsets of a topological space $X$ ordered by inclusion.

$\Delta: \mathbf{Set} \rightarrow [\mathcal{O}(X)^{op}, \mathbf{Set}]$

Write $\mathcal{P}$ for the presheaf functor category, and $P \in \mathcal{P}$ for the functor which maps $\mathcal{O}(X)^{op}$ to $\mathbf{Set}$. Take open sets $U, V$, such that $U \subseteq V$ in $X$. A presheaf consists of

\begin{itemize}

\item restriction maps, $P(V) \rightarrow P(U)$, these are morphisms which enforce some sort of ordering of the mapped sets,

\item and the actual mapped sets $P(U), P(V)$ which are called sections.

\end{itemize}

Since the question specifies a constant presheaf, by definition, the restriction maps of $\Delta A$ are identity maps. And the sections are just the $A$.
Specifically $\Delta A (U) = A$ for subsets $U$ of $X$, and $\Delta A (\rightarrow) = 1_A$ for morphisms.

Write $\Gamma P = P(X)$ for the \textbf{global} sections functor which takes an element of $\mathcal{P}$ to a $\mathbf{Set}$.

We are required to show a bijection:

For $A$ in $\mathbf{Set}$ and $B$ in $\mathcal{P}$
\begin{align*}
  \mathbf{Set}(A, \Gamma B) &\rightarrow \mathcal{P}(\Delta A, B)\\
\end{align*}
and
\begin{align*}
   \mathcal{P}(\Delta A, B) &\rightarrow \mathbf{Set}(A, \Gamma B)
\end{align*}

The maps between the presheaf functors in $\mathcal{P}$ are natural transformations. Natural transformations are a collection of maps $\alpha_A$: $\{ \Delta A (A) \rightarrow B(A) \}_{A \in \mathcal{A}}$. For $U \subseteq V \subseteq X$ we have the commuting square:
\begin{equation}
\label{eqn:13291}
\begin{tikzcd}
  & \Delta A(X) \arrow[d,"\alpha_X"] \arrow[r, "1_A"] & \Delta A(V) \arrow[d,"\alpha_{V}"] \arrow[r, "1_A"] & \Delta A(U) \arrow[d,"\alpha_{U}"{name=G,right}]\\
  & B(X) \arrow[r,"B(f)"] & B(V) \arrow[r,"B(f)"] & B(U)
\end{tikzcd}
\end{equation}

Recall $\Delta A (\cdot) = A$. Then the morphism in $\mathbf{Set}$ is represented by $\alpha_X$ above. As visible from the figure above this corresponds one to one with each $\alpha_A$ in $\mathcal{A}$, so the bijection holds. Dually using the exact same reasoning $\Pi$, the left adjoint of $\Delta$ is the presheaf evaluation at the empty set, $\Pi (P) = P(\emptyset)$.

For the left adjoint to $\Pi$, $\Lambda$, and for $A$ in $\mathbf{Set}$ and $B$ in $\mathcal{P}$, we need to show a bijection between:

\begin{equation*}
  \mathcal{P}(\Lambda A, B) \leftrightarrow \mathbf{Set}(A, \Pi(B))
\end{equation*}

To try and cobble together a definition of the presheaf functor $\Lambda$, start with the naturality diagram representing morphisms in $\mathcal{P}$:

\begin{center}
\begin{tikzcd}
  & \Lambda(U) \arrow[d,"\alpha_U"] \arrow[r, "A(f)"] & \Lambda(\emptyset) \arrow[d,"\alpha_{\emptyset}"{name=G,right}]\\
  & B(U) \arrow[r,"B(f)"] & B(\emptyset)
\end{tikzcd}
\end{center}

Note that $\Pi (B) = B(\emptyset)$. Start by choosing $\Lambda(\emptyset) = A$, so the morphism in $\mathbf{Set}$ is $\alpha_\emptyset$. Our choice of $\Lambda$ needs to make this diagram commute for all $U$ in $\mathcal{O}(X)^{op}$. For $U \neq \emptyset$ we could try $\Lambda(U) = A$, however to force the square above to commute with this choice, will impose some structure on the presheaf $B$. Rather, try setting  $\Lambda(U) = \emptyset$ for $U \neq \emptyset$. Choosing the initial object $\emptyset$ of $\mathcal{O}(X)^{op}$, means there is one map out of the top LHS of the square in the above diagram, and the square commutes as required.

We also have

\begin{equation*}
  \mathcal{P}(A, \nabla B) \leftrightarrow \mathbf{Set}(\Gamma A, B)
\end{equation*}

$\nabla$, the right adjoint to $\Gamma$ can be obtained dually, by swapping $\mathbf{Set}$ with $\mathbf{Set}^{op}$ and $\mathcal{O}(X)^{op}$ with $\mathcal{O}(X)$. This reverses the left and right adjoints.

\begin{equation*}
  \mathcal{P}(A, \nabla B) \leftrightarrow \mathbf{Set}(\Gamma A, B)
\end{equation*}

\end {document}