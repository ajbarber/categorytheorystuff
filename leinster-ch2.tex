\documentclass{article}
\usepackage{amsmath}
\usepackage{dsfont}
\usepackage{tikz-cd}
\usetikzlibrary{matrix}
\usepackage{lipsum}
\usepackage{titlesec}
\usepackage{parskip}
\begin{document}

\title{Leinster - Basic Category Theory - Selected problem solutions for Chapter 2}
\author{Adam Barber}

\maketitle
\subsubsection*{2.1.16}
\paragraph{(a)}

Interesting adjoint functors to $G$-sets.

The trivial group functor $I$ sends a set to a $\mathbf{G}$-set with the trivial action $gx = x$. Interesting functors

Orbit functor sends a $G$-set with underlying set elements $a$ of $A$ to:

$A_G = \{ g \cdot a, g \in G \}$

Fixed point functor sends a $G$-set  with underlying set elements $a$ of $A$ to:

$A^G = \{ a \text{ such that } g \cdot a = a \text{ for all } g \in G, a \in A \}$

\paragraph{Fixed point functor - right adjoint}

Morphisms in a $G$-set are functions on the underlying set, where $f$ commutes with $g$ for every $g \in G$.

There is a bijection for each $A \in \mathbf{Set}$ and $B \in [G, \mathbf{Set}]$ as follows

\begin{align*}
  [G, \mathbf{Set}](I(A), B) &\rightarrow \mathbf{Set}(A, B^G) \\
  \psi &\mapsto \overline{\psi}
\end{align*}

$\overline{\psi}$ sends each element $a$ of $A$ to $\psi(a)$ if $g \cdot a = a$, otherwise it sends $a$ to $\psi(\emptyset)$.

\begin{align*}
  \mathbf{Set}(A, B^G) &\rightarrow [G, \mathbf{Set}](I(A), B) \\
  \phi &\mapsto \overline{\phi}
\end{align*}

$\phi$ sends each $a \in A$ in the underlying set of the $G$-set to the $G$-set $(g, \overline{\phi}(a)), g \in G$.

\paragraph{Orbit functor - left adjoint}

There is a bijection for each $A \in [G, \mathbf{Set}]$ and $B \in \mathbf{Set}$ as follows

\begin{align*}
  \mathbf{Set}(A_G, B) &\rightarrow [G, \mathbf{Set}](A, I(B)) \\
  \psi &\mapsto \overline{\psi}
\end{align*}

So each morphism in $\mathbf{Set}$ sends the set formed by the orbits of an element $a$ of $A$, call this $a_G$, to $\psi(a_G)$, where $\psi$ is a function of sets.
Choose a $G$-set morphism $\overline{\psi} = \psi$, where $\overline{\psi}$ commutes with $g$ for every $g$ in $G$.

\begin{align*}
  [G, \mathbf{Set}](A, I(B)) &\rightarrow \mathbf{Set}(A_G, B)\\
  \phi &\mapsto \overline{\phi}
\end{align*}

Choose $\overline{\phi}$ to be a disjoint union of each orbit of $a$ in $A$, $\overline{\phi}(a)$ =  $\amalg\{\phi(g \cdot a), g \in G\}$

\subsubsection*{2.1.17}

Write $\mathcal{O}(X)$ for the poset of open subsets of a topological space $X$ ordered by inclusion.

$\Delta: \mathbf{Set} \rightarrow [\mathcal{O}(X)^{op}, \mathbf{Set}]$

Write $\mathcal{P}$ for the presheaf functor category, and $P \in \mathcal{P}$ for the functor which maps $\mathcal{O}(X)^{op}$ to $\mathbf{Set}$. Take open sets $U, V$, such that $U \subseteq V$ in $X$. A presheaf consists of

\begin{itemize}

\item restriction maps, $P(V) \rightarrow P(U)$, these are morphisms which enforce some sort of ordering of the mapped sets,

\item and the actual mapped sets $P(U), P(V)$ which are called sections.

\end{itemize}

Write $\Gamma P = P(X)$ for the \textbf{global} sections functor which takes an element of $\mathcal{P}$ to a $\mathbf{Set}$.

\textbf {Intuition} Defining the sections functor to map the entire space in this way means in $\mathbf{Set}$ we have a map into every section. You can think of the presheaf as loosely a blueprint on how to sort the sections. So if we provide a presheaf sorting some of these sections, and a mapping of these sections, then we can get back a new presheaf sorting the mapped sections.

For $A$ in $\mathbf{Set}$ and $B$ in $\mathcal{P}$
\begin{align*}
  \mathbf{Set}(A, \Gamma B) &\rightarrow \mathcal{P}(\Delta A, B)\\
  \phi &\mapsto \phi \circ P
\end{align*}

In $\mathcal{P}$ we take $\phi \circ P$ as the presheaf functor.

\end {document}