\documentclass{article}
\usepackage{amsmath}
\usepackage{dsfont}
\usepackage{tikz-cd}
\usetikzlibrary{matrix}
\usepackage{lipsum}
\usepackage{titlesec}
\usepackage{parskip}
\begin{document}

\title{Leinster - Basic Category Theory - Selected problem solutions for Chapter 4}
\author{Adam Barber}

\maketitle

\subsubsection*{4.1.27}

$H_A$ is naturally isomorphic if and only if $\alpha_A: H_A(B) \rightarrow H_{A^\prime}(B)$ is isomorphic for all $B$ in $\mathcal{A}$.

The naturality square of $H_A$ is constructed below.
For every map $g: B^\prime \rightarrow B$, $B, B^\prime$ in $\mathcal{A}$, the following square commutes

\begin{equation*}
\begin{tikzcd}
  & H_A(B) \arrow[d,"\alpha_A"] \arrow[r, "H_A(g) = \mathit{-} \circ g"] & H_A(B^\prime) \arrow[d,"\alpha_{A^\prime}"] \\
  & H_{A^\prime}(B) \arrow[r,"H_{A^\prime}(g) = \mathit{-} \circ g"] & H_{A^\prime}(B^\prime)
\end{tikzcd}
\end{equation*}

Moreover, since $H_A \cong H_{A^\prime}$,  $\alpha_A$ is an isomorphism for every $B$ in $\mathcal{A}$.

Now consider the square below. For an arbitrary $B$ in $\mathcal{A}$ we need to show there is a bijection between the component $\alpha_A$ and a morphism $\overline{f}$ in $\mathcal{A}$.

\begin{equation*}
\begin{tikzcd}
  & \mathcal{A}(B,A) \arrow[r, "\alpha_A = f \circ \mathit{-}"] & \mathcal{A}(B,A^\prime) \arrow[d,"(\mathit{-})(B)"] \\
  & A \arrow[u,"(\mathit{-})^B"] \arrow[r,"\overline{f}"] & A^\prime
\end{tikzcd}
\end{equation*}

But $(\mathit{-})^B$ and $(\mathit{-})(B)$ are unique and inverses so take $f,\overline{f}\colon A \rightarrow A^\prime$ as $\overline{f}(A) = f(A^B)(B) = A^\prime$ and  we have our required bijection, hence $A$ and $A^\prime$ are isomorphic. Note that the isomorphism of $\alpha_A$ must hold for all $B$ in $\mathcal{A}$. To see why, suppose there exists a $B$, such that $\alpha_A$ is a bijective morphism but not an isomorphism, then $\overline{f}$ is not an isomorphism, and we have a contradiction. Suppose, alternatively, there exists a $B$, such that $\alpha_A$ is not bijective, then our expression for $\overline{f}$ implies $\overline{f}$ is not bijective, and again we have a contradiction.

\subsubsection*{4.1.27 - another less convoluted attempt}

With $f\colon A \rightarrow B$, $A, B \in \mathcal{A}$ consider the following diagram

\begin{equation*}
\begin{tikzcd}
  & H_A(B) \arrow[d,"\alpha_B"] \arrow[r, "\mathit{-} \circ f"] & H_A(A) \arrow[d,"\alpha_{A}"] \\
  & H_{B}(B) \arrow[r,"\mathit{-} \circ f"] & H_{B}(A)
\end{tikzcd}
\end{equation*}

We require $g \circ f = 1_A$ and $f \circ g = 1_B$.

Set $g\colon B \rightarrow A$. By naturality above square commutes, so
\begin{align*}
  \alpha_A(g \circ f) &= (\alpha_B \circ g)(f),   \\
  \alpha_A(g \circ f) &= 1_B(f), \\
  g \circ f &= \alpha_A^{-1}1_Bf = 1_{H_A(A)} = 1_A, \\
\end{align*}
the last equality just being a matter of notation. The other direction proceeds analogously.

\subsubsection*{4.1.27 - even shorter version}

$H_A \cong H_{A^\prime}$. Both sides are functors from $\mathcal{A}^{op}$ to $\mathbf{Set}$. Functors preserve identity, so $H_A(A) = 1_{H_A(A)} = 1_A$, and similarly $H_{A^\prime}(A^\prime) = 1_{H_{A^\prime}(A^\prime)} = 1_{A^\prime}$. So $1_A \cong 1_{A^\prime}$.

\subsubsection*{4.1.28}

Here we construct a bijection between the set $U_p(G)$ and a group homomorphism $\phi$.

\begin{equation*}
\begin{tikzcd}
  & U_p(G) \arrow[d,"\alpha_G"] \arrow[r, "h"] & U_p(H) \arrow[d,"\alpha_{H}"] \\
  & \mathbf{Grp}(\mathds{Z}/p\mathds{Z}, G) \arrow[r,"h"] & \mathbf{Grp}(\mathds{Z}/p\mathds{Z}, H)
\end{tikzcd}
\end{equation*}

$U_p(G)$ is the set of $\{g \in G\colon g^p = 1\}$.

For the present question, take an arbitrary $g$ in $G$. Set $\phi(1) = g$. By the properties of a homormorphism we shall see this maps the additive group $\mathds{Z}/p\mathds{Z}$ to $U_p(G)$. $\phi$ preserves the identity so $\phi(0) = 1$. Since $\phi(1 + 1) = g^2$, generally $\phi(n) = g^n$. So $\phi(p) = g^p = \phi(0) = 1$. So $\phi$ maps to a group with order $p$, or simply order $1$ if $g$ is the element of the trivial group. So $\mathds{Z}/p\mathds{Z}$ sees groups of order $p$ or 1. This result means we have the required bijection, $\alpha$ and $\alpha^{-1}$ in the diagram above. Observing the diagram we just need to specify how morphisms in $\mathbf{Grp}(\mathds{Z}/p\mathds{Z}, -)$ work. They are simply group homomorphisms $h$, that take $\mathbf{Grp}(\mathds{Z}/p\mathds{Z}, G)$ to $\mathbf{Grp}(\mathds{Z}/p\mathds{Z}, H)$. So referring to the diagram, naturality holds, and we can conclude $\mathbf{Grp}(\mathds{Z}/p\mathds{Z}, -)$ and $U_p$ are naturally isomorphic. \footnote{A well known result is as follows. For a group homomorphism $\psi: G_1 \rightarrow G_2$, let $g \in G_1$ be of finite order $n$. Then $\psi(g)$ divides the order of $g$. Because $g^n = e_1$ implies $\psi(g)^n = \psi(g^n) = \psi(e_1) = e_2$. So if $p$ is prime then the resulting homomorphism maps to a group of order $p$ or 1.}

\subsubsection*{4.1.29}

Here we show a natural isomorphism between $\mathbf{CRing} \rightarrow \mathbf{Set}$ and $\mathbf{CRing}(\mathds{Z}[x],\mathit{-})$. Let $R$ be a ring, and $h\colon R \rightarrow S$ be a ring homomorphism.

\begin{equation*}
\begin{tikzcd}
  & U(R) \arrow[d,"\alpha_R"] \arrow[r, "h"] & U(S) \arrow[d,"\alpha_S"] \\
  & \mathbf{CRing}(\mathds{Z}[x], R) \arrow[r,"h"] & \mathbf{CRing}(\mathds{Z}[x], S)
\end{tikzcd}
\end{equation*}

For $\alpha_R$, we only require the \textbf{elements} of $r \in U(R)$ to construct the ring homomorphism from $\mathds{Z}[x]$ to $R$.

For $\alpha_R^{-1}$, we simply forget the ring structure by applying $U$.

From (0.13) we have there exists a unique ring homomorphism $\phi\colon \mathds{Z}[x] \rightarrow R$ such that $\phi(x) = r$. One can observe this ring homomorphism is analogous to the group homormorphism in (4.1.28), it is completely determined by the choice of $\phi(x)$. So the maps $\mathbf{CRing}(\mathds{Z}[x], R)$ are essentially the same as elements of $R$, and the description of the bijection is complete. The only thing remaining is to verify naturality, which here boils down to the morphisms $h$ being the same, whether acting on sets or rings.

\subsubsection*{4.1.30}

Let $X$ be a topological space, and $S$ be the Sierpinski space.

\begin{equation*}
\begin{tikzcd}
  & O(X) \arrow[d,"\alpha_X"] \arrow[r, "\mathcal{O}(g)"] & \mathcal{O}(Y) \arrow[d,"\alpha_Y"] \\
  & (f\colon X \rightarrow S) \arrow[r,"\mathit{-}\circ g"] & (fg\colon Y \rightarrow S)
\end{tikzcd}
\end{equation*}

\textbf{Note:} A function $f$ from one topological space $X$ into another topological space $S$ is continuous if and only if for every open set $V$ in $S$, $f^{-1}(V)$ is open in $X$.

Let $f$ be the continous map $f\colon X \rightarrow S$, let the singleton open set in the Sierpinski space $S$ be $V$. Since $f$ is continuous, we retrieve the open sets of $X$ by setting $\alpha_X^{-1} = f^{-1}(V)$.

In the $\alpha_X$ direction take the open sets of $X$, $\overline{X} = \mathcal{O}(X)$ and map them to $V$. So $\alpha_X = \overline{X} \mapsto (\overline{X} \rightarrow V)$.

To prove $\mathcal{O} \cong H_S$, we need to show the isomorphism is natural, specifically the square above commutes. But in the category $Top$ the morphisms are defined as the continous functions, so there is a one to one correspondence between $\mathcal{O}(g)$ and $\mathit{-} \circ g$, and the square indeed commutes.

\subsubsection*{4.1.32}

The naturality square of the composite functors, with $f: A^\prime \rightarrow A$, and $g: B \rightarrow B^\prime, p\colon A\rightarrow B$

\begin{equation*}
\begin{tikzcd}
  & \mathcal{B}(F(A),B) \arrow[d,"\alpha_{A,B}"] \arrow[r, "g \circ \mathit{-} \circ f"] & \mathcal{B}(F(A^\prime),B^\prime) \arrow[d,"\alpha_{A^\prime,B^\prime}"] \\
  & \mathcal{A}(A,G(B)) \arrow[r,"g \circ \mathit{-} \circ f"] & \mathcal{A}(A^\prime,G(B^\prime))
\end{tikzcd}
\end{equation*}


Maps in $\mathcal{B}(F(\mathit{-}),\mathit{-})$, represented by the top row, are
\begin{equation}
\label{eqn:41321}
F(A) \xrightarrow{Ff} F(A^\prime)\xrightarrow{q} B\xrightarrow{g} B^\prime.
\end{equation}

Maps in $\mathcal{A}(\mathit{-},G(\mathit{-}))$, represented by the bottom row, are
\begin{equation}
\label{eqn:41322}
  A \xrightarrow{f} A^\prime \xrightarrow{p} G(B) \xrightarrow{Gg} G(B^\prime).
\end{equation}

If the above square is a natural isomorphism, then the $\alpha$ are invertible for all $f, g$, and $p$. Equivalently the maps (\ref{eqn:41321}) and (\ref{eqn:41322}) are one to one.

In the other direction, if $F$ and $G$ are adjoint, then the $\alpha$ are one to one, for all $f, g$, by definition of adjointness (2.1). Also (\ref{eqn:41321}) and (\ref{eqn:41322}) are one to one by the result of Exercise 2.1.14. Hence natural isomorphism between $\mathcal{B}(F(\mathit{-}),\mathit{-})$ and $\mathcal{A}(\mathit{-},G(\mathit{-}))$ follows.

\subsubsection*{4.2.3}

Definition of functor $M^{op} \rightarrow \mathbf{Set}$. There is a single object in $M$, the underlying set of the group. So the functor is determined by its effect on morphisms. \textbf{Note: }Morphisms in the group category are simply elements of the group. So the functor is described as $F(m)(x) = x \cdot m$. The contravariant nature of the functor is convention, because we are right applying the group action, rather than left.

\paragraph{(a)} Since $M$ has only one object, then every other functor out of $M$, is isomorphic to $F(m)$ above. So the $F(m)$ is the unique representable functor, up to isomorphism.

\paragraph{(b)} Set $\alpha\colon \underline{M}  \rightarrow X$ to be $\alpha(1) = x \cdot m$ for all $m \in M$. Morphisms of $G$-sets preserve the group action, but have the equivariant property. Specifically, for $\alpha$, we require $\alpha(x) \cdot g = \alpha (x \cdot g)$, for all $x, g$. We have
\begin{equation}
\label{eqn:4231}
\begin{aligned}
  \alpha(m) &= \alpha(1) \cdot m \\
  \alpha(m^2) &= \alpha(1) \cdot m^2 \\
  \alpha(m^3) &= \alpha(1) \cdot m^3 \\
  \; \vdots
\end{aligned}
\end{equation}
So fixing $\alpha(1) = x$ fully determines the $M$-set map, which is unique. To see it is unique assume there is another map $\beta(1) = x$, then apply (\ref{eqn:4231}) analogously to recover $\beta = \alpha$.

\subsubsection*{4.3.15}

\paragraph{(a)}
$J(f)$ is an isomorphism so for $A, B \in \mathcal{A}$, $J(f)J(g) = 1_{J(B)}$, and $J(g)J(f) = 1_{J(A)}$. By the functor laws $J(f)J(g)=J(fg)=1_{J(B)}=J(1_B) \iff fg = 1_B$, for a given $A, B \in \mathcal{A}$. The expression for $gf$ proceeds analogously.

\paragraph{(b)} Follows from (a) and full and faithfulness.

\paragraph{(c)} If objects $A$ and $A^\prime$ are isomorphic, then there are maps $f\colon A \rightarrow A^\prime$, and $g\colon A^\prime \rightarrow A$ such that $fg = 1_A$, $gf=1_{A^\prime}$. The existence of $f, g$ imply isomorphisms $J(f), J(g)$ in $\mathcal{B}$ by (a), and hence objects $J(A), J(A^\prime)$ in $\mathcal{B}$ which are isomorphic. The other direction proceeds similarly.

\subsubsection*{4.3.16}

\paragraph{(a)}

Need to show that for $f \rightarrow H_f$, $f \neq g \implies H_f \neq H_g$. $H_f$ is defined as $p \mapsto f \circ p$, where $H_A(B) = p$. Consider $H_f$ at $B=A$, then $p=1_A$ and if $f \neq g$, indeed $H_f(1_A) \neq H_g(1_A)$

\paragraph{(b)}
Take $f: A \rightarrow A^{\prime}$. We need to show for every $H_f$, there is an inducing $f$.

With $g: B \rightarrow A$ in the following naturality diagram, observe that by naturality of $H_A$, $f$ is one to one with $H_f$.

\begin{equation*}
\begin{tikzcd}
  & H_A(A) \arrow[d,"H_f(1_A) = f"] \arrow[r, "\mathit{-} \circ g"] & H_A(B) \arrow[d,"H_f"] \\
  & H_{A^\prime}(A) \arrow[r,"\mathit{-} \circ g"] & H_{A^\prime}(B))
\end{tikzcd}
\end{equation*}

\paragraph{(c)}

The universal quality of $X\colon \mathcal{A^{\text{op}}} \rightarrow \textbf{Set}$ states a representation of $X$ consists of an object $A \in \mathcal{A}$, together with an  element $u \in X(A)$, such that
\textit{for each $B$ in $\mathcal{A}$, there is a unique map $\overline{x}\colon \mathcal{B} \rightarrow \mathcal{A}$ such that $X\overline{x}(u) = x$.}  So fixing $A$, and taking $\overline{x}=f\colon \mathcal{B} \rightarrow \mathcal{A}$, maps $f$ are in bijection with maps $Xf$. And maps $f$ are in bijection with $H_f$. The isomorphism between $H_A$ and $X$ is made clear by the following diagram. For a $A^\prime \in \mathcal{A}$:

\begin{equation*}
\begin{tikzcd}
  & H_A(A^\prime) \arrow[d, <->] \arrow[r, "H_f"] & H_{B}(A^\prime) \arrow[d, <->] \\
  & A \arrow[d, <->] \arrow[r,"f"] & B \arrow[d, <->] \\
  & X(A) \arrow[r, "Xf", {description}] & X(B)
\end{tikzcd}
\end{equation*}

Noting the both $H_A$ and $X$ are contravariant functors.

\subsubsection*{4.3.18}

\paragraph{(a)}

We are given $J\colon \mathcal{C} \rightarrow \mathcal{D}$ is fully faithful. We need to show $\mathit{-} \mapsto J \circ \mathit{-}$ is fully faithful. In a functor category this means that for a given pair of functors $F, F^\prime$ in $[\mathcal{B,C}]$,  the natural transformation in $[\mathcal{B,C}]$ is in one to one correspondence with the natural transformation between $J \circ F$ and $J \circ F^\prime$ in $[\mathcal{B},\mathcal{D}]$. The proof is best illustrated graphically:

Let $f$ be a morphism in $\mathcal{B}$.

\begin{tikzcd} [sep = .5 cm]
& & & F(B) \arrow [dl, dotted, red] \arrow [rr, "Ff"] \arrow [dd,"\alpha{B}"] & & F^\prime(B) \arrow [dl, dotted, red, "J"] \arrow [dd, "\alpha_{B^\prime}"] \\
& & J \circ F(B) \arrow [rr, "JFf" {description, right}, crossing over] & & J \circ F(B^\prime) \\
& & & F^\prime(B) \arrow [dl, dotted, red] \arrow [rr,"F^\prime f" {description}] & & F^\prime(B) \arrow [dl, dotted, red] \\
& & J\circ F^\prime(B) \arrow [rr, "JF^\prime f" {description, right}] \arrow [from = uu, crossing over] & & J \circ F(B^\prime) \arrow [from = uu, crossing over]\\
\end{tikzcd}

In the figure above, the mapping $Ff$, $JFf$ are one to one, by full and faithfulness of $J$. By the same reasoning, the rest of the mappings in the foremost square are one to one their respective mapping in the square behind. The diagram shows the one to one relationship between natural transformations between $F$  and $F^\prime$, and $J\circ F$ and $J \circ F^\prime$.
\paragraph{(b)}

Follows from Lemma 4.3.8 (c).

\paragraph{(c)}

We have $F \dashv G$, and $F \dashv G^\prime$. So $\mathcal{A}(A, G(B)) \cong \mathcal{B}(F(A),B) \cong \mathcal{A}(A,G^\prime(B))$, naturally in $A, B$.  Considering  the morphisms in $\mathcal{A}$, this can be written $H_{GB} \cong H_{G^\prime B}$, naturally in $A \in \mathcal{A}, B \in \mathcal{B}$. Alternatively this can be expressed as an application of the Yoneda embedding, $H_\bullet$. Specifically $H_{\bullet}\circ G(B) \cong H_{\bullet} \circ G^\prime(B)$, for all $B \in \mathcal{B}$, naturally.\footnote{Here $H_\bullet$ encapsulates the naturality in $A$ of this isomorphism} So we have $H_{\bullet} \circ G \cong H_{\bullet} \circ G^\prime$, and by (b) $G \cong G^\prime$.

\end {document}
