\documentclass{article}
\usepackage{amsmath}
\usepackage{dsfont}
\usepackage{tikz-cd}
\usetikzlibrary{matrix}
\usepackage{lipsum}
\usepackage{titlesec}
\usepackage{parskip}
\begin{document}

\title{Leinster - Basic Category Theory - Selected problem solutions for Chapter 3}
\author{Adam Barber}

\maketitle

\subsubsection*{4.1.27}

Take the naturality square of $H_A$, and construct with $A, A^\prime$ in the following way.
For every map $f: A \rightarrow A^\prime$, the following square commutes.

\begin{equation*}
\begin{tikzcd}
  & H_A(A) \arrow[d,"\alpha_A = f \circ \mathit{-}"] \arrow[r, "H_A(f) = \mathit{-} \circ f"] & H_A(A^\prime) \arrow[d,"\alpha_{A^\prime} = f \circ \mathit{-}"] \\
  & H_{A^\prime}(A) \arrow[r,"H_{A^\prime}(f) = \mathit{-} \circ f"] & H_{A^\prime}(A^\prime)
\end{tikzcd}
\end{equation*}

Moreover, since $H_A \cong H_{A^\prime}$,  $\alpha_A$ is an isomorphism for every $A$ in the codomain of $f$. Consider the left hand side component $\alpha_A$, which represents a map of $\mathbf{1} \rightarrow H_{A^\prime}(A)$. Now a map of $\mathbf{1}$ into $\mathcal{A}$ is the same thing as an object of $\mathcal{A}$, so we take $\alpha_A = \mathcal{A}(A,A^\prime)$, that is $\alpha_A$ represents an isomorphism in $\mathcal{A}$, and $A \cong A^\prime$.

\subsubsection*{4.1.28}

Here we construct a bijection between the set $U_p(G)$ and a group homomorphism $\phi$.

\begin{equation*}
\begin{tikzcd}
  & U_p(G) \arrow[d,"\alpha_G"] \arrow[r, "h"] & U_p(H) \arrow[d,"\alpha_{H}"] \\
  & \mathbf{Grp}(\mathds{Z}/p\mathds{Z}, G) \arrow[r,"h"] & \mathbf{Grp}(\mathds{Z}/p\mathds{Z}, H)
\end{tikzcd}
\end{equation*}

$U_p(G)$ is the set of $\{g \in G\colon g^p = 1\}$.

For the present question, take an arbitrary $g$ in $G$. Set $\phi(1) = g$. By the properties of a homormorphism we shall see this maps the additive group $\mathds{Z}/p\mathds{Z}$ to $U_p(G)$. $\phi$ preserves the identity so $\phi(0) = 1$. Since $\phi(1 + 1) = g^2$, generally $\phi(n) = g^n$. So $\phi(p) = g^p = \phi(0) = 1$. So $\phi$ maps to a group with order $p$, or simply order $1$ if $g$ is the element of the trivial group. So $\mathds{Z}/p\mathds{Z}$ sees groups of order $p$ or 1. This result means we have the required bijection, $\alpha$ and $\alpha^{-1}$ in the diagram above. Observing the diagram we just need to specify how morphisms in $\mathbf{Grp}(\mathds{Z}/p\mathds{Z}, -)$ work. They are simply group homomorphisms $h$, that take $\mathbf{Grp}(\mathds{Z}/p\mathds{Z}, G)$ to $\mathbf{Grp}(\mathds{Z}/p\mathds{Z}, H)$. So referring to the diagram, naturality holds, and we can conclude $\mathbf{Grp}(\mathds{Z}/p\mathds{Z}, -)$ and $U_p$ are naturally isomorphic. \footnote{A well known result is as follows. For a group homomorphism $\psi: G_1 \rightarrow G_2$, let $g \in G_1$ be of finite order $n$. Then $\psi(g)$ divides the order of $g$. Because $g^n = e_1$ implies $\psi(g)^n = \psi(g^n) = \psi(e_1) = e_2$. So if $p$ is prime then the resulting homomorphism maps to a group of order $p$ or 1.}


\end {document}
