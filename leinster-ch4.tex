\documentclass{article}
\usepackage{amsmath}
\usepackage{dsfont}
\usepackage{tikz-cd}
\usetikzlibrary{matrix}
\usepackage{lipsum}
\usepackage{titlesec}
\usepackage{parskip}
\begin{document}

\title{Leinster - Basic Category Theory - Selected problem solutions for Chapter 4}
\author{Adam Barber}

\maketitle

\subsubsection*{4.1.27}

$H_A$ is naturally isomorphic if and only if $\alpha_A: H_A(B) \rightarrow H_{A^\prime}(B)$ is isomorphic for all $B$ in $\mathcal{A}$.

The naturality square of $H_A$ is constructed below.
For every map $g: B^\prime \rightarrow B$, $B, B^\prime$ in $\mathcal{A}$, the following square commutes

\begin{equation*}
\begin{tikzcd}
  & H_A(B) \arrow[d,"\alpha_A"] \arrow[r, "H_A(g) = \mathit{-} \circ g"] & H_A(B^\prime) \arrow[d,"\alpha_{A^\prime}"] \\
  & H_{A^\prime}(B) \arrow[r,"H_{A^\prime}(g) = \mathit{-} \circ g"] & H_{A^\prime}(B^\prime)
\end{tikzcd}
\end{equation*}

Moreover, since $H_A \cong H_{A^\prime}$,  $\alpha_A$ is an isomorphism for every $B$ in $\mathcal{A}$.

Now consider the square below. For an arbitrary $B$ in $\mathcal{A}$ we need to show there is a bijection between the component $\alpha_A$ and a morphism $\overline{f}$ in $\mathcal{A}$.

\begin{equation*}
\begin{tikzcd}
  & \mathcal{A}(B,A) \arrow[r, "\alpha_A = f \circ \mathit{-}"] & \mathcal{A}(B,A^\prime) \arrow[d,"(\mathit{-})(B)"] \\
  & A \arrow[u,"(\mathit{-})^B"] \arrow[r,"\overline{f}"] & A^\prime
\end{tikzcd}
\end{equation*}

But $(\mathit{-})^B$ and $(\mathit{-})(B)$ are unique and inverses so take $f,\overline{f}\colon A \rightarrow A^\prime$ as $\overline{f}(A) = f(A^B)(B) = A^\prime$ and  we have our required bijection, hence $A$ and $A^\prime$ are isomorphic. Note that the isomorphism of $\alpha_A$ must hold for all $B$ in $\mathcal{A}$. To see why, suppose there exists a $B$, such that $\alpha_A$ is a bijective morphism but not an isomorphism, then $\overline{f}$ is not an isomorphism, and we have a contradiction. Suppose, alternatively, there exists a $B$, such that $\alpha_A$ is not bijective, then our expression for $\overline{f}$ implies $\overline{f}$ is not bijective, and again we have a contradiction.

\subsubsection*{4.1.27 - another less convoluted attempt}

With $f\colon A \rightarrow B$, $A, B \in \mathcal{A}$ consider the following diagram

\begin{equation*}
\begin{tikzcd}
  & H_A(B) \arrow[d,"\alpha_B"] \arrow[r, "\mathit{-} \circ f"] & H_A(A) \arrow[d,"\alpha_{A}"] \\
  & H_{B}(B) \arrow[r,"\mathit{-} \circ f"] & H_{B}(A)
\end{tikzcd}
\end{equation*}

We require $g \circ f = 1_A$ and $f \circ g = 1_B$.

Set $g\colon B \rightarrow A$. By naturality above square commutes, so
\begin{align*}
  \alpha_A(g \circ f) &= (\alpha_B \circ g)(f),   \\
  \alpha_A(g \circ f) &= 1_B(f), \\
  g \circ f &= \alpha_A^{-1}1_Bf = 1_{H_A(A)} = 1_A, \\
\end{align*}
the last equality just being a matter of notation. The other direction proceeds analogously.

\subsubsection*{4.1.27 - even shorter version}

$H_A \cong H_{A^\prime}$. Both sides are functors from $\mathcal{A}^{op}$ to $\mathbf{Set}$. Functors preserve identity, so $H_A(A) = 1_{H_A(A)} = 1_A$, and similarly $H_{A^\prime}(A^\prime) = 1_{H_{A^\prime}(A^\prime)} = 1_{A^\prime}$. So $1_A \cong 1_{A^\prime}$.

\subsubsection*{4.1.28}

Here we construct a bijection between the set $U_p(G)$ and a group homomorphism $\phi$.

\begin{equation*}
\begin{tikzcd}
  & U_p(G) \arrow[d,"\alpha_G"] \arrow[r, "h"] & U_p(H) \arrow[d,"\alpha_{H}"] \\
  & \mathbf{Grp}(\mathds{Z}/p\mathds{Z}, G) \arrow[r,"h"] & \mathbf{Grp}(\mathds{Z}/p\mathds{Z}, H)
\end{tikzcd}
\end{equation*}

$U_p(G)$ is the set of $\{g \in G\colon g^p = 1\}$.

For the present question, take an arbitrary $g$ in $G$. Set $\phi(1) = g$. By the properties of a homormorphism we shall see this maps the additive group $\mathds{Z}/p\mathds{Z}$ to $U_p(G)$. $\phi$ preserves the identity so $\phi(0) = 1$. Since $\phi(1 + 1) = g^2$, generally $\phi(n) = g^n$. So $\phi(p) = g^p = \phi(0) = 1$. So $\phi$ maps to a group with order $p$, or simply order $1$ if $g$ is the element of the trivial group. So $\mathds{Z}/p\mathds{Z}$ sees groups of order $p$ or 1. This result means we have the required bijection, $\alpha$ and $\alpha^{-1}$ in the diagram above. Observing the diagram we just need to specify how morphisms in $\mathbf{Grp}(\mathds{Z}/p\mathds{Z}, -)$ work. They are simply group homomorphisms $h$, that take $\mathbf{Grp}(\mathds{Z}/p\mathds{Z}, G)$ to $\mathbf{Grp}(\mathds{Z}/p\mathds{Z}, H)$. So referring to the diagram, naturality holds, and we can conclude $\mathbf{Grp}(\mathds{Z}/p\mathds{Z}, -)$ and $U_p$ are naturally isomorphic. \footnote{A well known result is as follows. For a group homomorphism $\psi: G_1 \rightarrow G_2$, let $g \in G_1$ be of finite order $n$. Then $\psi(g)$ divides the order of $g$. Because $g^n = e_1$ implies $\psi(g)^n = \psi(g^n) = \psi(e_1) = e_2$. So if $p$ is prime then the resulting homomorphism maps to a group of order $p$ or 1.}

\subsubsection*{4.1.29}

Here we show a natural isomorphism between $\mathbf{CRing} \rightarrow \mathbf{Set}$ and $\mathbf{CRing}(\mathds{Z}[x],\mathit{-})$. Let $R$ be a ring, and $h\colon R \rightarrow S$ be a ring homomorphism.

\begin{equation*}
\begin{tikzcd}
  & U(R) \arrow[d,"\alpha_R"] \arrow[r, "h"] & U(S) \arrow[d,"\alpha_S"] \\
  & \mathbf{CRing}(\mathds{Z}[x], R) \arrow[r,"h"] & \mathbf{CRing}(\mathds{Z}[x], S)
\end{tikzcd}
\end{equation*}

For $\alpha_R$, we only require the \textbf{elements} of $r \in U(R)$ to construct the ring homomorphism from $\mathds{Z}[x]$ to $R$.

For $\alpha_R^{-1}$, we simply forget the ring structure by applying $U$.

From (0.13) we have there exists a unique ring homomorphism $\phi\colon \mathds{Z}[x] \rightarrow R$ such that $\phi(x) = r$. One can observe this ring homomorphism is analogous to the group homormorphism in (4.1.28), it is completely determined by the choice of $\phi(x)$. So the maps $\mathbf{CRing}(\mathds{Z}[x], R)$ are essentially the same as elements of $R$, and the description of the bijection is complete. The only thing remaining is to verify naturality, which here boils down to the morphisms $h$ being the same, whether acting on sets or rings.

\end {document}
